\section{Canonical Projection}\label{sec:canonical projection}

By inserting projectors to sectors of conserved quantum numbers we can evaluate canonical-ensemble expectation values via grand canonical expectation values.
For example, since the total fermion number $N$ and the spin along $h$ have definite value in a canonical sector,
\begin{align}
    \expectation{\O}_{N, S_h}
    =
    \frac{\tr{ e^{-\beta H} \O}_{N, S_h}}{\tr{ e^{-\beta H}}_{N, S_h}}
    =
    \frac{\tr{ e^{-\beta (H-\mu \hat{N}-h \cdot \hat{S})} \O }_{N, S_h}}{\tr{ e^{-\beta (H-\mu \hat{N}-h \cdot \hat{S})}}_{N, S_h}} 
    =
    \frac{\tr{ e^{-\beta (H-\mu\hat{N}-h\cdot \hat{S})} P_N P_{S_h} \O}}{\tr{ e^{-\beta (H - \mu \hat{N} - h \cdot \hat{S})} P_N P_{S_h}}}
    =
    \frac{\expectation{P_N P_{S_h} \O}}{\expectation{P_N P_{S_h}}}
    \label{eq:canonical-grand canonical projection}
\end{align}
where the $N$ and $S_h$ expectation-value and trace subscripts indicate the desired sector, $\hat{N}$ and $\hat{S}$ the number and spin operators, $P_N$ and $P_{S_h}$ the operator that project to the $N$-body sector with definite spin along $h$, and subscript-free traces and expectation values are grand-canonical.

For spin-half fermions these projection operators can be written in an operator form,
\begin{align}
    P_N &= \oneover{2\Volume+1} \sum_{n=0}^{2\Volume} \exp\left(i \phi_n (\hat{N}-N)\right)
    &
    \phi_n &= \frac{2\pi n}{2\Volume+1}
    &
    N &\in \left\{0, 1, \cdots, 2\Volume\right\}
    \\
    P_{S_h} &= \oneover{2\Volume+1} \sum_{s=-\Volume}^{\Volume} \exp\left(i \phi_s (2 \hat{h}\cdot \hat{S} - 2 S_h )\right)
    &
    \phi_s &= \frac{2\pi s}{2\Volume+1}
    &
    S_h &\in \half \left\{ -\Volume, \cdots, 0, \cdots, +\Volume\right\}
\end{align}
Since $[\hat{N}, \hat{S}]=0$ we can combine these projectors together,
\begin{align}
    \PP_{N, S_h} &= \oneover{(2\Volume+1)^2} \sum_{n,s} \exp\left(\frac{2\pi i}{2\Volume+1}\left[ n (\hat{N}-N) + s (2\hat{h}\cdot\hat{S} - 2 S_h)\right]\right)
\end{align}
So, we should understand how to Trotterize the identity \eqref{canonical-grand canonical projection} and transform the projected grand-canonical traces into an auxiliary-field equivalents.

Actually, transforming the Trotterized projected grand-canonical traces proceeds in a straightfoward way that matches \eqref{trotterization}, \eqref{trotterization bilinearized}, and the derivation of the fermion matrix \secref{fermion matrix}, but with an extra Grassman `timeslice' that gives matrix elements of the projector,
\begin{align}
    \matrixElement{\eta}{ \PP_{N, S_h} }{\eta'}
    =&
    \matrixElement{\eta}{
        \oneover{(2\Volume+1)^2}\sum_{n,s} \exp\left(\frac{2\pi i}{2\Volume+1}\left[ n \left(\sum_{ijx}\psi\adjoint_{ix} \delta_{ij}\psi_{jx} -N\right) + s \left(2 \hat{h}^k \sum_{xij}\psi\adjoint_{ix} \half \sigma_{ij}^k \psi_{jx} - 2S_h\right)\right]\right)
    }{\eta'}
    \nonumber\\
    =&
    \oneover{(2\Volume+1)^2}\sum_{ns} \exp\left(-\frac{2\pi i}{2\Volume+1} (nN+2s S_h)\right)\times
    \nonumber\\
    &\matrixElement{\eta}{
        \exp\left(\frac{2\pi i}{2\Volume+1}\left[ n \left(\sum_{ijx}\psi\adjoint_{ix} \delta_{ij}\psi_{jx}\right) + s \left(2 \hat{h}^k \sum_{xij}\psi\adjoint_{ix} \half \sigma_{ij}^k \psi_{jx}\right)\right]\right)
    }{\eta'}
    \nonumber\\
    =&
    \oneover{(2\Volume+1)^2}\sum_{ns} \exp\left(-\frac{2\pi i}{2\Volume+1} (nN+2s S_h)\right)\times
    \nonumber\\
    &\matrixElement{\eta}{
        \exp\left(\frac{2\pi i}{2\Volume+1}\left[ \left(\sum_{ijx}\psi\adjoint_{ix}\left[n \delta_{ji} + s \hat{h}^k\sigma_{ji}^k\right]\psi_{jx}\right) \right]\right)
    }{\eta'}
    \nonumber\\
    =&
    \oneover{(2\Volume+1)^2}\sum_{ns} \exp\left(-\frac{2\pi i}{2\Volume+1} (nN+2s S_h)\right)\times
    \exp\left[
        \sum_{ijx}
        \eta\adjoint_{ix}
            \exp\left( \frac{2\pi i}{2\Volume+1} \left(n\one + s \hat{h}\cdot \sigma\right) \right)_{ji}
        \eta'_{jx}
    \right]
\end{align}
This matrix element appears as an additional term in the Grassman integral \eqref{grassmann integral} with an accompanying diagonal piece.
The projected Trotterized trace therefore yields a sum of expressions just like the standard fermion determinant \eqref{fermion determinant with 2nt timeslices}, giving a fermion matrix analogous to $\DD$ \eqref{DD}, each with a different phase factor and an extended sausage,
\begin{align}
    \tr{\PP_{N,S_h} \prod_{t=1}^{N_t} e^{-\Delta t K}e^{\psi\adjoint\left(A_t + \Delta t \mu + \Delta t h\cdot\sigma\right)\psi}}
    =
    \oneover{(2\Volume+1)^2} \sum_{ns} e^{-\frac{2\pi i}{2\Volume+1} (nN+2sS_h)}\; \det\left[ \one + \UU \exp\left\{ \frac{2\pi i}{2\Volume+1}\left(n + s \hat{h} \cdot \sigma\right)\right\}\right].
\end{align}
Using \href{https://en.wikipedia.org/wiki/Determinant#Sylvester's\_determinant\_theorem}{Sylvester's identity} we can put the projection factor on the left of $\UU$.

So the projected grand-canonica auxiliary field partition function is not given by \eqref{computational partition function}, but we can multiply by 1 to get the expected action and evaluate the reweighting factor
\begin{align}
    \expectation{\PP_{N, S_h}}
    =
    \expectation{
        \oneover{(2\Volume+1)^2} \sum_{ns} e^{-\frac{2\pi i}{2\Volume+1} (nN+2sS_h)}\; \frac{
            \det\left[ \one + \exp\left\{ \frac{2\pi i}{2\Volume+1}\left(n + s \hat{h} \cdot \sigma\right)\right\} \UU\right]
        }{
            \det\left[ \one + \UU \right]
        }
    }
    \label{eq:grand canonical projector representation}
\end{align}
That is, if we can write the expectation value in a form like the computational partition function \eqref{computational partition function},
\begin{align}
    Z_{N, S_h} = \expectation{\PP_{N,S_h}} &= \int DA\; e^{-S} \PP_{N, S_h} = \int DA\; \oneover{(2\Volume+1)^2}\sum_{ns} e^{-\frac{2\pi i}{2\Volume+1}(nN+2sS_h)} e^{-S_{ns}}
    \\
    S_{ns} &= \half \sum_{t} A_t\transpose (-\Delta\tilde{t}\tilde{V})\inverse A_t - \log\det\left(\one+\exp\left\{\frac{2\pi i}{2\Volume+1}(n\one+s \hat{h}\cdot\sigma)\right\}\UU\right) + \frac{N_t}{2} \trlog{ -2\pi \Delta\tilde{t}\tilde{V}}
\end{align}
then if the grand canonical $\expectation{\O}$ for some operator $\O$ can be formulated as a functional derivative of $Z$ we may evaluate canonical expectation value \eqref{canonical-grand canonical projection} via the same derivative of $Z_{N, S_h}$,
\begin{align}
    \expectation{\O}_{N,S_h} = \partial \log Z_{N, S_h}
    &= \frac{\int DA\; \oneover{(2\Volume+1)^2}\sum_{ns} e^{-\frac{2\pi i}{2\Volume+1}(nN+2sS_h)} e^{-S_{ns}} \O_{ns}}{\int DA\; \oneover{(2\Volume+1)^2}\sum_{ns} e^{-\frac{2\pi i}{2\Volume+1}(nN+2sS_h)} e^{-S_{ns}}}
    \nonumber\\
    &= \frac{\int DA\; e^{-S} \oneover{(2\Volume+1)^2}\sum_{ns} e^{-\frac{2\pi i}{2\Volume+1}(nN+2sS_h)} e^{S-S_{ns}} \O_{ns}}{\int DA\; e^{-S}\oneover{(2\Volume+1)^2}\sum_{ns} e^{-\frac{2\pi i}{2\Volume+1}(nN+2sS_h)} e^{S-S_{ns}}}
    \nonumber\\
    &= \frac{\expectation{\oneover{(2\Volume+1)^2}\sum_{ns} e^{-\frac{2\pi i}{2\Volume+1}(nN+2sS_h)} e^{S-S_{ns}} \O_{ns}}}{\expectation{\oneover{(2\Volume+1)^2}\sum_{ns} e^{-\frac{2\pi i}{2\Volume+1}(nN+2sS_h)} e^{S-S_{ns}}}}
    = \frac{\expectation{\PP_{N,S_h} \O_{ns}}}{\expectation{\PP_{N,S_h}}}
\end{align}
and we conclude that and $\expectation{\PP_{N,S_h}}$ is proportional to the canonical partition function in that sector.
So, the (absolute) free energy is out of bounds, but functional derivatives are OK.
However, remember that we \emph{functionally differentiate first} and then multiply by unity to ensure the grand canonical Boltzmann factor is used as the computational measure, and that we need to calculate the observable in each Fourier component.
The Fourier-dependent observable can be computed with the same method as in the grand-canonical case but promoting $\UU \goesto \exp\{2\pi i(n+s\hat{h}\cdot\sigma)/(2\Volume+1)\}\UU$.

In fact, looking at the explicit form of $\UU$ \eqref{fugacity}, we can think of $S_{ns}$ as the normal action but with the replacements
\begin{align}
    \tilde{\mu}&\goesto \tilde{\mu} + \frac{2\pi i}{2\Volume+1} \frac{n}{\tilde{\beta}}
    &
    \tilde{\vec{h}} \goesto \tilde{\vec{h}} + \frac{2\pi i}{2\Volume+1} \frac{2s}{\tilde{\beta}} \hat{h}
\end{align}
Then, to find observables we can functionally differentiate $\exp(-S_{ns})$ ``under the sum'', and only then insert $\det(\one+\UU)$ both upstairs and downstairs to get the canonically-projected grand-canonical evaluation of the observable.

\subsection{The $\beta\goesto0$ Limit}\label{sec:canonical infinite temperature}

In a canonical sector with $N$ particles and total spin $S_h$ we can evaluate the partition function $Z_{N,S_h}$ at high temperature as in \secref{infinite temperature}, since the gaussian still approaches a Dirac delta.
So, we need to evaluate
\begin{align}
    \expectation{\PP_{N,S_h}}
    \goesto&
    \oneover{(2\Volume+1)^2}\oneover{\det(\one+\UU)} \sum_{ns} e^{-\frac{2\pi i}{2\Volume+1}(nN+2sS_h)} \det(\one+\UU_{ns})
    \nonumber\\
    =&
    \oneover{(2\Volume+1)^2}\oneover{\det(\two)} \sum_{ns} e^{-\frac{2\pi i}{2\Volume+1}(nN+2sS_h)} \det\left((1+e^{+\frac{2\pi i}{2\Volume+1}(n + s \hat{h}\cdot\sigma)})\one\right)
    \nonumber\\
    =&
    \oneover{(2\Volume+1)^2}\oneover{2^{2\Volume}} \sum_{ns} e^{-\frac{2\pi i}{2\Volume+1}(nN+2sS_h)} \left(1+e^{\frac{2\pi i}{2\Volume+1}(n+s)}\right)^\Volume \left(1+e^{\frac{2\pi i}{2\Volume+1}(n-s)}\right)^\Volume 
    \nonumber\\
    =&
    \oneover{(2\Volume+1)^2}\oneover{2^{2\Volume}} \sum_{nsab} e^{-\frac{2\pi i}{2\Volume+1}(nN+2sS_h)} \begin{pmatrix} \Volume \\ a \end{pmatrix} e^{\frac{2\pi i}{2\Volume+1} a (n+s)} \begin{pmatrix} \Volume \\ b \end{pmatrix} e^{\frac{2\pi i}{2\Volume+1} b (n-s)}
    \nonumber\\
    =&
    \oneover{(2\Volume+1)^2}\oneover{2^{2\Volume}} \sum_{ab} \begin{pmatrix} \Volume \\ a \end{pmatrix} \begin{pmatrix} \Volume \\ b \end{pmatrix} \sum_n e^{\frac{2\pi i}{2\Volume+1}n(a+b-N)} \sum_s e^{\frac{2\pi i}{2\Volume+1}s(a-b-2S_h)}
    \nonumber\\
    =&
    \oneover{2^{2\Volume}} \sum_{ab} \begin{pmatrix} \Volume \\ a \end{pmatrix} \begin{pmatrix} \Volume \\ b \end{pmatrix} \delta_{a+b,N} \delta_{a-b,2S_h}
    \nonumber\\
    =&
    \oneover{2^{2\Volume}} \begin{pmatrix} \Volume \\ \frac{N}{2} + S_h \end{pmatrix} \begin{pmatrix} \Volume \\ \frac{N}{2}-S_h \end{pmatrix}
    \label{eq:canonical infinite-temperature partition function}
\end{align}
which can also be computed by counting all states with $\frac{N}{2} + S_h$ spin-\up\ particles and $\frac{N}{2} - S_h$ spin-\dn\ particles.

\subsection{Exact Calculations in the $N=2$, $S_h=0$ Sector}

One purpose of going through this whole rigamarole is to check the code's correctness.
If we pick a small enough lattice and a sector with few enough particles we can evaluate expectation values according to the Trotterization \eqref{trotterization} projected to a canonical sector.
Since the Hubbard-Stratonovich transformation and subsequent manipulations are exact, we can make an apples-to-apples comparison of observables at finite $\Delta \tilde{t}$ given enough statistics.

Let us proceed by evaluating the kinetic and potential in a two-body state
\begin{align}
    \ket{a,b} = \alpha \tilde{\psi}\adjoint_{\up a} \tilde{\psi}\adjoint_{\dn b} \ket{0}
\end{align}
where $\ket{0}$ is the Fock vacuum and $\alpha$ is a normalization which we now calculate (up to a phase).
Requiring \ket{a,b} to be normalized, and leveraging the anticommutation relations,
\begin{align}
    1\equiv\braket{a,b}{a,b} 
    &= \abs{\alpha}^2 \matrixElement{0}{\tilde{\psi}_{\dn b}\tilde{\psi}_{\up a} \tilde{\psi}\adjoint_{\up a} \tilde{\psi}\adjoint_{\dn b}}{0}
    \nonumber\\
    &= \abs{\alpha}^2 \matrixElement{0}{\tilde{\psi}_{\dn b}(1-\tilde{\psi}\adjoint_{\up a} \tilde{\psi}_{\up a}) \tilde{\psi}\adjoint_{\dn b}}{0}
    = \abs{\alpha}^2 \matrixElement{0}{1-\tilde{\psi}\adjoint_{\dn b} \tilde{\psi}_{\dn b}}{0}
    = \abs{\alpha}^2 \braket{0}{0},
\end{align}
so we pick $\alpha=1$ for convenience.
A similar calculation shows that we have
\begin{align}
    \braket{c,d}{a,b} = \delta_{ac}\delta_{bd}
\end{align}

Next, let us evaluate the (dimensionless) two-body kinetic energy; the momentum-squared piece of the dimensionless Hamiltonian \eqref{dimensionless hamiltonian}.
\begin{align}
    \tilde{K}_{a'b';ab} = \matrixElement{a',b'}{
        \sum_{cd} \tilde{\psi}\adjoint_c \tilde{\kappa}_{cd} \tilde{\psi}_d
    }{a,b}
\end{align}
where the diagonal sum over spins is suppressed.
Writing out the state, and applying the anticommuation relations until ladder operators destroy the Fock vacuum,
\begin{align}
    \tilde{K}_{a'b';ab}
    &= \matrixElement{0}{
            \tilde{\psi}_{b'\dn} \tilde{\psi}_{a'\up} 
            \left(\sum_{cd} \tilde{\psi}\adjoint_c \tilde{\kappa}_{cd} \tilde{\psi}_d\right)
            \tilde{\psi}\adjoint_{a\up} \tilde{\psi}\adjoint_{b\dn}
        }{0}
    \nonumber\\
    &= \sum_{cd\sigma} \tilde{\kappa}_{cd} \matrixElement{0}{
            \tilde{\psi}_{b'\dn} \tilde{\psi}_{a'\up} 
            \tilde{\psi}\adjoint_{c\sigma} \tilde{\psi}_{d\sigma}
            \tilde{\psi}\adjoint_{a\up} \tilde{\psi}\adjoint_{b\dn}
        }{0}
    \nonumber\\
    &= \sum_{cd\sigma} \tilde{\kappa}_{cd} \matrixElement{0}{
            \tilde{\psi}_{b'\dn}
            \left(\delta_{a'c}\delta_{\up\sigma} - \tilde{\psi}\adjoint_{c\sigma} \tilde{\psi}_{a'\up} \right)
            \left(\delta_{ad}\delta_{\up\sigma} - \tilde{\psi}\adjoint_{a\up}\tilde{\psi}_{d\sigma} \right)
            \tilde{\psi}\adjoint_{b\dn}
        }{0}
    \nonumber\\
    &= \tilde{\kappa}_{a'a} \delta_{b'b} + \sum_{cd} \tilde{\kappa}_{cd} \matrixElement{0}{
            \tilde{\psi}_{b'\dn}
            \tilde{\psi}\adjoint_{c\dn} \tilde{\psi}_{a'\up}
            \tilde{\psi}\adjoint_{a\up}\tilde{\psi}_{d\dn}
            \tilde{\psi}\adjoint_{b\dn}
        }{0}
    \nonumber\\
    &= \tilde{\kappa}_{a'a} \delta_{b'b} + \sum_{cd} \tilde{\kappa}_{cd} \matrixElement{0}{
            \left(\delta_{b'c} - \tilde{\psi}\adjoint_{c\dn}\tilde{\psi}_{b'\dn} \right)
            \left(\delta_{a'a} - \tilde{\psi}\adjoint_{a\up}\tilde{\psi}_{a'\up} \right)
            \left(\delta_{db} - \tilde{\psi}\adjoint_{b\dn} \tilde{\psi}_{d\dn} \right)
        }{0}
    \nonumber\\
    &= \tilde{\kappa}_{a'a} \delta_{b'b} + \tilde{\kappa}_{b'b} \delta_{a'a}.
    \label{eq:two-body kinetic energy}
\end{align}
Of course: the kinetic energy is just the sum of the kinetic energy of each particle.

Finally, we evaluate the (dimensionless) two-body potential energy; the remaining piece of the dimensionless Hamiltonian \eqref{dimensionless hamiltonian}
\begin{align}
    \tilde{V}_{a'b';ab}/\Volume
    =& \matrixElement{a',b'}{
        -\sum_{cd} \tilde{\psi}\adjoint_c
            \frac{\tilde{C}_0}{2} \delta_{cd}
            \tilde{\psi}_d
        + \frac{1}{2} \sum_{cd} \tilde{n}_c \tilde{V}_{cd} \tilde{n}_d
    }{a,b}
    \nonumber\\
    =&
    -\sum_c \half \tilde{C}_0 \matrixElement{a',b'}{n_c}{a,b}
    + \frac{1}{2} \sum_{cd} \tilde{C}_R \LegoSphere{R}_{cd} \matrixElement{a',b'}{\tilde{n}_c \tilde{n}_d}{a,b}
    \nonumber\\
    =&
    -\sum_{c\sigma} \half \tilde{C}_0 \matrixElement{0}{\tilde{\psi}_{b'\dn} \tilde{\psi}_{a'\up} n_{c\sigma} \tilde{\psi}\adjoint_{a\up}\tilde{\psi}\adjoint_{b\dn} }{0}
    + \frac{1}{2} \sum_{cd\sigma\tau} \tilde{C}_R \LegoSphere{R}_{cd} \matrixElement{0}{\tilde{\psi}_{b'\dn} \tilde{\psi}_{a'\up} \tilde{n}_{c\sigma} \tilde{n}_{d\tau} \tilde{\psi}\adjoint_{a\up}\tilde{\psi}\adjoint_{b\dn}}{0}
    \nonumber\\
    =&
    -\sum_{c\sigma} \half \tilde{C}_0 \matrixElement{0}{\tilde{\psi}_{b'\dn} \tilde{\psi}_{a'\up} \left(\delta_{ac}\delta_{\sigma\up}\tilde{\psi}\adjoint_{c\sigma} + \tilde{\psi}\adjoint_{a\up} n_{c\sigma}\right)\tilde{\psi}\adjoint_{b\dn}}{0}
    \nonumber\\
    &+ \frac{1}{2} \sum_{cd\sigma\tau} \tilde{C}_R \LegoSphere{R}_{cd} \matrixElement{0}{\tilde{\psi}_{a'\up} \tilde{\psi}_{b'\dn} \tilde{n}_{c\sigma} \left(\delta_{ad}\delta_{\tau\up}\tilde{\psi}\adjoint_{d\tau} +  \tilde{\psi}\adjoint_{a\up}\tilde{n}_{d\tau}\right) \tilde{\psi}\adjoint_{b\dn}}{0}
    \nonumber\\
    =&
    - \half \tilde{C}_0 \delta_{a'a}\delta_{b'b}
    -\sum_{c\sigma} \half \tilde{C}_0 \matrixElement{0}{\tilde{\psi}_{b'\dn} \tilde{\psi}_{a'\up} \tilde{\psi}\adjoint_{a\up} \left(\delta_{cb}\delta_{\sigma\dn}\tilde{\psi}\adjoint_{c\sigma} + \tilde{\psi}\adjoint_{b\dn}n_{c\sigma}\right)}{0}
    \nonumber\\
    &+ \frac{1}{2} \sum_{cd\sigma\tau} \tilde{C}_R \LegoSphere{R}_{cd} \left[
        \delta_{ad}\delta_{\tau\up} \matrixElement{a'b'}{\tilde{n}_{c\sigma}}{d,b}
        +\matrixElement{0}{\tilde{\psi}_{a'\up} \tilde{\psi}_{b'\dn} \tilde{n}_{c\sigma} \tilde{\psi}\adjoint_{a\up}\left(\delta_{bd}\delta_{\dn\tau} \tilde{\psi}\adjoint_{d\tau} + \tilde{\psi}\adjoint_{b\dn}\tilde{n}_{d\tau}\right)}{0}
        \right]
    \nonumber\\
    =&
    - \tilde{C}_0 \delta_{a'a}\delta_{b'b}
    + \half \sum_{c\sigma} \tilde{C}_R \LegoSphere{R}_{ca} \matrixElement{a',b'}{\tilde{n}_{c\sigma}}{a,b}
    + \frac{1}{2} \sum_{c\sigma} \tilde{C}_R \LegoSphere{R}_{cb}
        \matrixElement{a',b'}{\tilde{n}_{c\sigma}}{a,b}
    \nonumber\\
    =&
    - \tilde{C}_0 \delta_{a'a}\delta_{b'b}
    + \half \sum_{c\sigma} \tilde{C}_R \left(\LegoSphere{R}_{ca} + \LegoSphere{R}_{cb}\right) \matrixElement{a',b'}{\tilde{n}_{c\sigma}}{a,b}
    \nonumber\\
    =&
    - \tilde{C}_0 \delta_{a'a}\delta_{b'b}
    + \half \sum_{c} \tilde{C}_R \left(\LegoSphere{R}_{ca} + \LegoSphere{R}_{cb}\right) \delta_{a'a}\delta_{b'b} \left(\delta_{ca} + \delta_{cb}\right)
    \nonumber\\
    =&
    - \tilde{C}_0 \delta_{a'a}\delta_{b'b}
    + \half \tilde{C}_R \left(\LegoSphere{R}_{aa} + \LegoSphere{R}_{bb} + \LegoSphere{R}_{ab} + \LegoSphere{R}_{ba}\right)\delta_{a'a}\delta_{b'b}
\end{align}
where the repeated indices on the Lego spheres are unsummed.  Since the spheres are symmetric, we can cancel the $\half$ and write
\begin{align}
    \tilde{V}_{a'b';ab}/\Volume
    =&
    \delta_{a'a}\delta_{b'b} \left(
    + \tilde{C}_R \LegoSphere{R}_{aa} + \tilde{C}_R \LegoSphere{R}_{ab}
    - \tilde{C}_0
    \right).
\end{align}
Note that $\LegoSphere{R}_{aa} = 0$ unless $R=0$.
So, if the on-site interaction is included $\tilde{C}_0\neq 0$ the subtracted piece exactly cancels the first term, while if no on-site interaction is included $\tilde{C}_0=0$ the subtracted piece vanishes, but so does the first piece, independently.
Therefore
\begin{align}
    \tilde{V}_{a'b';ab} = \Volume \delta_{a'a} \delta_{b'b} \; \tilde{C}_R \LegoSphere{R}_{ab},
    \label{eq:two-body potential energy}
\end{align}
the obvious answer.

With the two-body kinetic \eqref{two-body kinetic energy} and potential \eqref{two-body potential energy} energies, we can construct the $N=2$, $S_h=0$ canonically-projected Trotterized trace,
\begin{align}
    \tr{ \PP_{2,0}  \prod_{t=1}^{N_t}
        e^{-\Delta \tilde{t}\tilde{K}}
        e^{-\Delta \tilde{t} \tilde{V} }
    } = \tr{ \prod_{t=1}^{N_t}
        e^{-\Delta \tilde{t}\tilde{K}}
        e^{-\Delta \tilde{t} \tilde{V} }
    }_{N=2, S_h=0}
\end{align}
needed to evaluate canonical expectation values \eqref{canonical-grand canonical projection} with $\mu=0$ and $\vec{h}=0$.
Since the values of $\mu$ and $\vec{h}$ only serve to numerically stabilize the canonically-projected auxiliary field partition function, this restriction does not influence canonically-projected expectation values.

By inserting the two-body representation of an observable into the trace and taking the ratio we can get exact results for finite-$\Delta\tilde{t}$ calculations done via Monte Carlo.
