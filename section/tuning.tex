\section{Tuning}\label{sec:tuning}

To tune our Hamiltonian we go to the two-body sector.
Let $\ket{x,y} = \psi\adjoint_\up(x) \psi\adjoint_\down(y) \ket{0}$.
Then
\begin{equation}
	(HM\Delta x^2) \ket{x,y} = \sum_{x'} k_{x,x'}\ket{x',y} + \sum_{y'} k_{y,y'}\ket{x,y'} + V_{xy} \ket{x,y}
\end{equation}
where we use a dimensionless potential $V_{xy}$ to generalize the contact interaction;
the obvious contact interaction in the Hamiltonian \eqref{hamiltonian} is
\begin{equation}
	V_{xy} = - gM \delta_{xy}
\end{equation}
and $V_{xy}=V_{yx}$ by Newton's 3rd law.

Let's apply the non-interacting Hamiltonian $H_0$ to two-body momentum states
\begin{equation}
	\ket{p,q} 
	=
	\oneover{\sqrt{V}^2}\sum_{xy} e^{i(p\cdot x + q \cdot y)} \ket{x,y}
	=
	\oneover{N_x^2}\sum_{xy} e^{i(p\cdot x + q \cdot y)} \ket{x,y}
\end{equation}
to find
\begin{align}
	(H_0 M\Delta x^2) \ket{p,q}
	&=
			\oneover{N_x^2} \sum_{xx'y} e^{i(p\cdot x + q \cdot y)} k_{xx'}\ket{x',y}
		+	\oneover{N_x^2} \sum_{xyy'} e^{i(p\cdot x + q \cdot y)} k_{yy'}\ket{x,y'}
	\\
	&=
			\oneover{N_x^2} \sum_{xx'p'y} e^{i(p\cdot x + q \cdot y)} \frac{p'^2}{2V} e^{-i p'\cdot(x-x')}\ket{x',y}
		+	\oneover{N_x^2} \sum_{xyy'q'} e^{i(p\cdot x + q \cdot y)} \frac{q'^2}{2V} e^{-i q'\cdot(y-y')}\ket{x,y'}
	\\
	&=
			\oneover{N_x^4} \sum_{x'p'y} \frac{p'^2}{2} e^{i(p'x+qy)} \ket{x',y} \sum_{x} e^{i(p-p')x}
		+	\oneover{N_x^4} \sum_{xy'q'} \frac{q'^2}{2} e^{i(px+q'y)} \ket{x,y'} \sum_{y} e^{i(q-q')y}
	\\
	&=
			\oneover{N_x^4} \sum_{x'p'y} \frac{p'^2}{2} e^{i(p'x'+qy)} \ket{x',y} V \delta_{pp'}
		+	\oneover{N_x^4} \sum_{xy'q'} \frac{q'^2}{2} e^{i(px+q'y')} \ket{x,y'} V \delta_{qq'}
	\\
	&=
			\frac{p^2}{2} \oneover{N_x^2}\sum_{x'y} e^{i(px'+qy)} \ket{x',y}
		+	\frac{q^2}{2} \oneover{N_x^2}\sum_{xy'} e^{i(px+qy')} \ket{x,y'}
	\\
	&=
		\left(\frac{p^2}{2} + \frac{q^2}{2}\right) \ket{p,q}.
\end{align}

Now consider the states with total momentum $P$ and a relative coordinate $r$,
\begin{align}
	\ket{P,r} = \oneover{N_x} \sum_x e^{iPx} \ket{x,x+r}
\end{align}
The applying the (interacting) Hamiltonian gives
\begin{align}
	(H M\Delta x^2) \ket{P,r}
	&=
			\sum_{r'} k_{rr'} \left( e^{iP(r'-r)}+1\right)\ket{P,r'}
		+	V_{0,r} \ket{P,r}
\end{align}
If we focus on $P=0$ we get
\begin{align}
	(H M\Delta x^2) \ket{0,r}
	&=
			\sum_{r'} 2k_{rr'} \ket{0,r'}
		+	V_{0,r} \ket{0,r};
\end{align}
the $2$ on the kinetic piece is the fact that the reduced mass is half the single-particle mass.
So, we can solve the Schr\"{o}dinger equation in the $P=0$ sector,
\begin{align}
	(H M\Delta x^2) \sum_r \ket{P=0,r}\braket{P=0,r}{\psi} &= (EM\Delta x^2)\ket{\psi}
	\\
	\sum_r \bra{P=0, r'} (H M \Delta x^2) \ket{P=0,r}\braket{P=0,r}{\psi} &= (E M\Delta x^2)\braket{P=0,r'}{\psi}
	\\
	h_{r'r} \psi_r &= e \psi_{r'}
\end{align}
Since this is in the two-body sector we can use the L\"{u}scher finite-volume quantization condition \eqref{quantization condition} to determine the desired continuum-limit energy levels and adjust $V$ until we reproduce them correctly.


\subsection{$D_4$ Symmetry; the $A_1$ representation}

The finite volume / Brillouin zone has the symmetry of a square, $D_4$, the group of 8 elements (the identity, 3 nontrivial rotations, 2 vertex and 2 edge reflections).  The `$S$-wave' is the $A_1$ representation.

We can project $h_{r'r}$ into the $A_1$ sector for a computational advantage while tuning.
