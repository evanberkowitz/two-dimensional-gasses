\section{Tuning}\label{sec:tuning}

To tune our Hamiltonian we go to the two-body sector.
Let $\ket{x,y} = \psi\adjoint_\up(x) \psi\adjoint_\down(y) \ket{0}$.
Then
\begin{equation}
    \tilde{H} \ket{x,y} = \sum_{x'} k_{x,x'}\ket{x',y} + \sum_{y'} k_{y,y'}\ket{x,y'} + V_{xy} \ket{x,y}
\end{equation}
where we use a dimensionless potential $V_{xy}$ to generalize the contact interaction;
the obvious contact interaction in the Hamiltonian \eqref{hamiltonian} is
\begin{equation}
    V_{xy} = - gM \delta_{xy}
\end{equation}
and $V_{xy}=V_{yx}$ by Newton's 3rd law.

Let's apply the non-interacting Hamiltonian $H_0$ to two-body momentum states
\begin{equation}
    \ket{p,q} 
    =
    \oneover{\sqrt{V}^2}\sum_{xy} e^{i(p\cdot x + q \cdot y)} \ket{x,y}
    =
    \oneover{N_x^2}\sum_{xy} e^{i(p\cdot x + q \cdot y)} \ket{x,y}
\end{equation}
to find
\begin{align}
    \tilde{H}_0 \ket{p,q}
    &=
            \oneover{N_x^2} \sum_{xx'y} e^{i(p\cdot x + q \cdot y)} k_{xx'}\ket{x',y}
        +   \oneover{N_x^2} \sum_{xyy'} e^{i(p\cdot x + q \cdot y)} k_{yy'}\ket{x,y'}
    \\
    &=
            \oneover{N_x^2} \sum_{xx'p'y} e^{i(p\cdot x + q \cdot y)} \frac{p'^2}{2V} e^{-i p'\cdot(x-x')}\ket{x',y}
        +   \oneover{N_x^2} \sum_{xyy'q'} e^{i(p\cdot x + q \cdot y)} \frac{q'^2}{2V} e^{-i q'\cdot(y-y')}\ket{x,y'}
    \\
    &=
            \oneover{N_x^4} \sum_{x'p'y} \frac{p'^2}{2} e^{i(p'x+qy)} \ket{x',y} \sum_{x} e^{i(p-p')x}
        +   \oneover{N_x^4} \sum_{xy'q'} \frac{q'^2}{2} e^{i(px+q'y)} \ket{x,y'} \sum_{y} e^{i(q-q')y}
    \\
    &=
            \oneover{N_x^4} \sum_{x'p'y} \frac{p'^2}{2} e^{i(p'x'+qy)} \ket{x',y} V \delta_{pp'}
        +   \oneover{N_x^4} \sum_{xy'q'} \frac{q'^2}{2} e^{i(px+q'y')} \ket{x,y'} V \delta_{qq'}
    \\
    &=
            \frac{p^2}{2} \oneover{N_x^2}\sum_{x'y} e^{i(px'+qy)} \ket{x',y}
        +   \frac{q^2}{2} \oneover{N_x^2}\sum_{xy'} e^{i(px+qy')} \ket{x,y'}
    \\
    &=
        \left(\frac{p^2}{2} + \frac{q^2}{2}\right) \ket{p,q}.
\end{align}

Now consider the states with total momentum $P$ and a relative coordinate $r$,
\begin{align}
    \ket{P,r} = \oneover{N_x} \sum_x e^{iPx} \ket{x,x+r}
\end{align}
The applying the (interacting) Hamiltonian gives
\begin{align}
    \tilde{H} \ket{P,r}
    &=
            \sum_{r'} k_{rr'} \left( e^{iP(r'-r)}+1\right)\ket{P,r'}
        +   V_{0,r} \ket{P,r}
\end{align}
If we focus on $P=0$ we get
\begin{align}
    \tilde{H} \ket{0,r}
    &=
            \sum_{r'} 2k_{rr'} \ket{0,r'}
        +   V_{0,r} \ket{0,r};
\end{align}
the $2$ on the kinetic piece is the fact that the reduced mass is half the single-particle mass.
So, we can solve the Schr\"{o}dinger equation in the $P=0$ sector,
\begin{align}
    \tilde{H} \sum_r \ket{P=0,r}\braket{P=0,r}{\psi} &= \tilde{E}\ket{\psi}
    \\
    \sum_r \bra{P=0, r'} \tilde{H} \ket{P=0,r}\braket{P=0,r}{\psi} &= \tilde{E}\braket{P=0,r'}{\psi}
    \\
    h_{r'r} \psi_r &= \tilde{E} \psi_{r'}
    \label{eq:two-body schrodinger equation}
\end{align}
Since this is in the two-body sector we can use the L\"{u}scher finite-volume quantization condition \eqref{quantization condition} to determine the desired continuum-limit energy levels and adjust $V_{0,r}$ until we reproduce them correctly.


\subsection{$D_4$ Symmetry; the $A_1$ representation}

The finite volume / Brillouin zone has the symmetry of a square, $D_4$, the group of 8 elements (the identity, 3 nontrivial rotations, 2 vertex and 2 edge reflections).  The `$S$-wave' is the $A_1$ representation.
It's the $A_1$ representation that satisfies the L\"{u}scher finite-volume quantization condition \eqref{quantization condition} (of course there are quantization conditions for the other irreps, and for $P\neq 0$, but we only have 1 parameter to tune, so let's tune it as simply as possible).

We can project $h_{r'r}$ into the $A_1$ sector for a computational advantage while tuning.
Consider transforming $\psi_{r'}$ into relative-momentum space,
\begin{align}
    \ket{p} = \oneover{\sqrt{V}}\sum_{r'} e^{ipr'} \psi_{r'}
\end{align}
In the free case the relative momentum $p$ is conserved and we can organize the energy eigenstates into momentum shells.
The momentum is given by an ordered pair of integers, and the size of the momentum shell is the number of unique ordered pairs in a single momentum vector's image under $D_4$.
For example, $(0,0)$ is a shell, $(0,\pm 1)$ and $(\pm 1, 0)$ form a shell of size 4, $(\pm 1, \pm 1)$ is a shell of size 4, $(\pm 1, \pm 2)$ and $(\pm 2, \pm 1)$ form a shell of size 8.
If we pick $N_x$ odd there's no funny business about double-counting the boundary.

The $A_1$-projected state is the equally-weighted sum of all the states in the shell with no funny phases,
\begin{align}
    \ket{A_1, \vec{p}} = \oneover{\shell{p}} \sum_{g\in D_4} \ket{g\vec{p}}
\end{align}
where $\shell{p}$ is the required normalization, given by $\sqrt{\textrm{the number of states in the shell}}$.
So, continuing our examples,
\begin{align}
    \ket{A_1, (0,0)} =& \ket{(0,0)}
    \\
    \ket{A_1, (0,1)} =& \frac{1}{2}\left(\ket{(0,+1)} + \ket{(0,-1)} + \ket{(+1,0)} + \ket{(-1,0)} \right)
    \\
    \ket{A_1, (1,1)} =& \frac{1}{2}\left(\ket{(+1,+1)} + \ket{(+1,-1)} + \ket{(-1,+1)} + \ket{(-1,-1)} \right)
    \\
    \ket{A_1, (1,2)} =& \frac{1}{\sqrt{8}}\big(\ket{(+1,+2)} + \ket{(+1,-2)} + \ket{(-1,+2)} + \ket{(-1,-2)} \nonumber\\
                     &\phantom{\frac{1}{\sqrt{8}}}+ \ket{(+2,+1)} + \ket{(+2,-1)} + \ket{(-2,+1)} + \ket{(-2,-1)} \big).
\end{align}

