\section{Effective Range Expansion and L\"{u}scher in 2D}

To understand what we mean by this theory we'll need to take some sort of continuum / infinite volume limit.
As in 3D this means tuning some scattering amplitude to match the target scattering data.

For partial wave labeled by angular momentum $\ell$ in 2D the effective range expansion reads
\begin{align}
	\cot \delta_\ell(p)
	&=
			\frac{2}{\pi} \log(p R_\ell)
		-	\frac{1}{a_\ell} p^{-2\ell}
		+	\frac{1}{2} r_\ell p^{2-2\ell}
		+	\cdots.
	\\
	\cot \delta(p)
	&=
			\frac{2}{\pi} \log(p R_0)
		-	\frac{1}{a_0}
		+	\frac{1}{2} r_0 p^2
\end{align}
where $R$ is an arbitrary dimensionful scale and (in the second line) the `scattering length' $a_0$ is actually dimensionless (note the units on the LHS) and $r_0$ has dimension +2.
All even-spatial-dimension effective range expansions get this log, and every non-log term gets an additional factor of $p^{2-D}$ which happens to vanish when $D=2$.
As a ressult, what's strange/magical about $D=2$ is that we can absorb $a_0$ into the first time by changing the scale in the log,
\begin{align}
	\tilde{a} &= R_0 e^{-\pi/2a_0}
	&
	\cot \delta(p)
	&=
			\frac{2}{\pi} \log(p \tilde{a})
		+	\frac{1}{2} r_0 p^2
		+	\cdots
	\label{eq:no-constant}
\end{align}
but we cannot further absorb the higher terms because they have a real, meaningful momentum dependence.
So, could we even mean by something like `unitary gas' in 2D?
It is not possible, for example, to have a momentum-independent phase shift.
The issue is that even if $r$ and all higher shape parameters vanish, the $\log(p\tilde{a})$ vanishes when its argument is $1$ and any momentum is allowed.
So if $\tilde{a}$ is fixed by the Hamiltonian then there is unavoidable momentum dependence.

Is $\tilde{a}$ what is meant by the 2D scattering length?  Or something else?
At the cold-atoms/nuclear matter ECT* program we saw experimental results always described as a function of $\log k_F a$ (I don't mean to suggest this $a$ is the above $\tilde{a}$ necessarily; I have no idea).
If we take our cue from this, maybe we'll see that we actually have to retune at each desired $k_F$, which would be a disaster.

Let's not forget that there is a \emph{geometric meaning} to the scattering length: where, with linear extrapolation, the wavefunction hits $0$.
According to \Refs{Adhikari:1986a,Adhikari:1986b,Khuri:2008ib,Galea:2017jhe}, we recover this geometric meaning when we pick
\begin{align}
	\cot \delta(p) = \frac{2}{\pi}\left[\log \frac{pa}{2} + \gamma\right] + \oneover{4} r_e^2 p^2 + \cdots
	\label{eq:ere}
\end{align}
where $\gamma = 0.577\ldots$ is the Euler–Mascheroni constant and $r_e$ has a similar geometric meaning (see \Ref{Galea:2017jhe} (18)).
We will call this expression \eqref{ere} `the' effective range expansion.
We can match this form to the form \eqref{no-constant} where we totally absorb the constant into $\tilde{a}$ if we pick
\begin{equation}
	\tilde{a} = \frac{e^{\gamma}}{2} a.
\end{equation}

Let us try to use the L\"{u}scher formula in 2D to tune.
In 2D we have
\begin{align}
	\cot \delta(p) - \frac{2}{\pi} \log \frac{pL}{2\pi} &= \frac{1}{\pi^2} S_2\left(\left(\frac{pL}{2\pi}\right)^2\right)
	&	
	S_2(x) &= \lim_{N\goesto\infty} \sum_{\abs{\vec{n}} \leq \frac{N}{2}} \oneover{n^2-x} - 2\pi \log \frac{N}{2}
	\label{eq:quantization condition}
\end{align}
In \Figref{S2} we show the zeta function $S_2$ which appears in the finite-volume quantization condition \eqref{quantization condition}.
Unlike in 3D we can evaluate this with good precision in the obvious way: just pick a high cutoff.
\begin{figure}
	\includegraphics[width=0.5\textwidth]{mma/S2.pdf}
	\caption{The 2D L\"{u}scher zeta function $S_2$.  We typically evaluate thinking of $x = (pL/2\pi)^2$; the non-interacting energies are shown as vertical asymptotes.}
	\label{fig:S2}
\end{figure}

So, let's take the ERE \eqref{ere} and try to apply the quantization condition \eqref{quantization condition} to learn how to tune,
\begin{align}
	\frac{2}{\pi} \left[\log\frac{pa}{2} + \gamma \right] + \cdots
	&=
	\frac{2}{\pi} \log \frac{pL}{2\pi} + \frac{1}{\pi^2} S_2\left(\left(\frac{pL}{2\pi}\right)^2\right)
	\\
	\log \frac{\pi a}{L} &= \oneover{2\pi} S_2\left(\left(\frac{pL}{2\pi}\right)^2\right) - \gamma
	\label{eq:renormalization condition}
\end{align}
where we dropped all further momentum dependence from the ERE.

If we come back to the tuning \eqref{tuning} and set $\Lambda=2\pi/\Delta x$ then we find
\begin{align}
	gM
	&= - \frac{2\pi}{\log\left(\frac{e^\gamma}{2} a \frac{2\pi}{\Delta x}\right)}
	= -\frac{2\pi}{\gamma + \log \pi a / \Delta x}.
\end{align}
where $a/\Delta x$ is the scattering length in units of the lattice spacing.
If we assume that we want a very large scattering length (on the scale of $L$) then we see that the log's argument $\sim N_x$.

Since `weak coupling' leads to energy levels given by the asymptotes in \Figref{S2}, the strong coupling region means trying to tune via \eqref{renormalization condition} to a constant where we get something between the asymptotes.
Which constant is somewhat a tricky question.
In 3D we tune to the zeroes of the zeta function; in 3D the best we can do is tune $a \sim L/\pi$.

The quantization condition \eqref{quantization condition} is phrased in terms of dimensionful variables $p$ and $L$,
\begin{align}
	\Delta x^2 p^2 &= 2 M E \Delta x^2
	&
	L &= N_x \Delta x
\end{align}
Let's translate these into dimensionless quantities.
\begin{align}
	\frac{p^2 L^2}{(2\pi)^2}
	=
	\frac{ \frac{(2 M E \Delta x^2)}{\Delta x^2} (N_x \Delta x)^2}{(2\pi^2)}
	=
	\frac{ 2 \tilde{E} N_x^2 }{(2\pi)^2}
\end{align}
where $\tilde{E}$ is the dimensionless energy eigenvalue of the dimensionless $\tilde{H}$ \eqref{hamiltonian}.
