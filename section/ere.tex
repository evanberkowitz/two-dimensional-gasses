\section{Effective Range Expansion and L\"{u}scher in 2D}

To understand what we mean by this theory we'll need to take some sort of continuum / infinite volume limit.
As in 3D this means tuning some scattering amplitude to match the target scattering data.

For partial wave labeled by angular momentum $\ell$ in 2D the effective range expansion reads
\begin{align}
	\cot \delta_\ell(p)
	&=
			\frac{2}{\pi} \log(p R_\ell)
		-	\frac{1}{a_\ell} p^{-2\ell}
		+	\frac{1}{2} r_\ell p^{2-2\ell}
		+	\cdots.
	\\
	\cot \delta(p)
	&=
			\frac{2}{\pi} \log(p R_0)
		-	\frac{1}{a_0}
		+	\frac{1}{2} r_0 p^2
\end{align}
where $R$ is an arbitrary dimensionful scale and (in the second line) the `scattering length' $a_0$ is actually dimensionless (note the units on the LHS) and $r_0$ has dimension +2.
All even-spatial-dimension effective range expansions get this log, and every non-log term gets an additional factor of $p^{2-D}$ which happens to vanish when $D=2$.
As a ressult, what's strange/magical about $D=2$ is that we can absorb $a_0$ into the first time by changing the scale in the log,
\begin{align}
	\tilde{a} &= R_0 e^{-\pi/2a_0}
	&
	\cot \delta(p)
	&=
			\frac{2}{\pi} \log(p \tilde{a})
		+	\frac{1}{2} r_0 p^2
		+	\cdots
		\label{eq:ere}
\end{align}
but we cannot further absorb the higher terms because they have a real, meaningful momentum dependence.
So, could we even mean by something like `unitary gas' in 2D?
It is not possible, for example, to have a momentum-independent phase shift.
The issue is that even if $r$ and all higher shape parameters vanish, the $\log(p\tilde{a})$ vanishes when its argument is $1$ and any momentum is allowed.
So if $\tilde{a}$ is fixed by the Hamiltonian then there is unavoidable momentum dependence.\footnote{
	2D is very special; could it be that the log gets handled when you try to transform $\delta$ into a scattering \emph{cross section}?
	One way to say when we have unitarity in 3D is to say that the scattering cross section saturates $\sigma_\ell = 4\pi k^{-2} (2\ell+1) \sin^2 \delta_l$.
	But certainly the $k^2$ downstairs is dependent on 3D; perhaps when we look at the crosss section in 2D there's a weird log?
	I think that would ALSO be bad, but I should admit that I'm confused.
	}

Is $\tilde{a}$ what is meant by the 2D scattering length?  Or something else?
At the cold-atoms/nuclear matter ECT* program we saw experimental results always described as a function of $\log k_F a$ (I don't mean to suggest this $a$ is the above $\tilde{a}$ necessarily; I have no idea).
If we take our cue from this, maybe we'll see that we actually have to retune at each desired $k_F$, which would be a disaster.

But before we fall too deep into despair, perhaps looking at the L\"{u}scher formula in 2D may shed light?
Then we might know how to tune?

In 2D we have
\begin{align}
	\cot \delta(p) - \frac{2}{\pi} \log \frac{pL}{2\pi} &= \frac{1}{\pi^2} S_2\left(\left(\frac{pL}{2\pi}\right)^2\right)
	&	
	S_2(x) &= \lim_{N\goesto\infty} \sum_{\abs{\vec{n}} \leq \frac{N}{2}} \oneover{n^2-x} - 2\pi \log \frac{N}{2}
	\label{eq:quantization condition}
\end{align}
In \Figref{S2} we show the zeta function $S_2$ which appears in the finite-volume quantization condition \eqref{quantization condition}.
\begin{figure}
	\includegraphics[width=0.5\textwidth]{mma/S2.pdf}
	\caption{The 2D L\"{u}scher zeta function $S_2$.  We typically evaluate thinking of $x = (pL/2\pi)^2$; the non-interacting energies are shown as vertical asymptotes.}
	\label{fig:S2}
\end{figure}

So, let's take the ERE \eqref{ere} and try to apply the quantization condition \eqref{quantization condition} to learn how to tune,
\begin{align}
	\frac{2}{\pi} \log(p \tilde{a}) + \cdots
	&=
	\frac{2}{\pi} \log \frac{pL}{2\pi} + \frac{1}{\pi^2} S_2\left(\left(\frac{pL}{2\pi}\right)^2\right)
	\\
	\frac{2}{\pi} \log \frac{2\pi \tilde{a}}{L} &= \oneover{\pi^2} S_2\left(\left(\frac{pL}{2\pi}\right)^2\right)
\end{align}
where we dropped all non-log dependence from the ERE.

In 3D to get unitarity we tune until we hit the zeros of $S_3(x)$ for some energies (one level per Lego Sphere operator we want to use).
In our case this suggests we would find
\begin{align}
	\frac{\tilde{a}}{L} = \frac{1}{2\pi},
\end{align}
meaning that if we set $R_0=L/2\pi$ (so that it's the radius of the compactified spatial directions, or equivalently one over the lowest nonzero allowed box momentum) then we would need the dimensionless $1/a_0 = 0$.
That seems promising?
But of course you can change $R_0$ and need a finite $a_0$.
So who knows.

If we come back to the tuning \eqref{tuning} and set $\Lambda=2\pi/\Delta x$ then we find
\begin{align}
	gM &= - \frac{2\pi}{\log\left(\frac{L}{2\pi} \frac{2\pi}{\Delta x}\right)} = -\frac{2\pi}{\log N_x}
\end{align}
to tune to the zeros of $S_2$.

Is any of this right for getting strong interactions?  I do not know.
