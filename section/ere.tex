\section{Effective Range Expansion 2D}

To understand what we mean by this theory we'll need to take some sort of continuum / infinite volume limit (since our Hamiltonian has no length scale in it, this should be reached by taking $N_x$ large).
As in 3D this means tuning some scattering amplitude to match the target scattering data.

For partial wave labeled by angular momentum $\ell$ in 2D the effective range expansion reads
\begin{align}
	\cot \delta_\ell(p)
	&=
			\frac{2}{\pi} \log(p R_\ell)
		-	\frac{1}{a_\ell} p^{-2\ell}
		+	\frac{1}{2} r_\ell p^{2-2\ell}
		+	\cdots.
	\\
	\cot \delta_0(p)
	&=
			\frac{2}{\pi} \log(p R_0)
		-	\frac{1}{a_0}
		+	\frac{1}{2} r_0 p^2
\end{align}
where $R$ is an arbitrary dimensionful scale and (in the second line) the `scattering length' $a_0$ is actually dimensionless (note the units on the LHS) and $r_0$ has dimension +2.
All even-spatial-dimension effective range expansions get this log, and every non-log term gets an additional factor of $p^{2-D}$ which happens to vanish when $D=2$.

In 2D the cross section (cross-length?) carries dimension of length and is given by
\begin{align}
	\lambda &= \frac{4}{k}\left(\sin^2 \delta_0 + 2\sum_{\ell=1}^{\infty} \sin^2 \delta_\ell \right);
\end{align}
see, for example, (2.14) of \Ref{Adhikari:1986a}.
Since $\sin^2 \delta = 1/(1+\cot^2\delta)$ we can express the cross section as
\begin{align}
	\lambda
	=
	\frac{4}{k}\left(\frac{1}{1+\cot^2 \delta_0} + 2\sum_{\ell=1}^{\infty} \frac{1}{1+\cot^2 \delta_\ell} \right)
	&=
	\frac{4}{k}\left(\frac{1}{1+\cot^2 \delta_0}\right) \left(1+ 2\sum_{\ell=1}^{\infty} \frac{1+\cot^2 \delta_0}{1+\cot^2 \delta_\ell} \right)
	\\
	&\approx
	\frac{4}{k}\left(\frac{1}{1+\cot^2 \delta_0}\right) \left(1+ 2\sum_{\ell=1}^{\infty} \order{k^{4\ell}} \right)
\end{align}
so at low momentum the $\ell=0$ scattering dominates the cross section.

What's strange/magical about $D=2$ is that we can absorb $a_0$ into the first time by changing the scale in the log,
\begin{align}
	a &= R_0 e^{-\pi/2a_0}
	&
	\cot \delta(p)
	&=
			\frac{2}{\pi} \log(p a)
		+	\sigma_2 p^2
		+	\cdots \text{(analytic in }p^2)
	\label{eq:ere}
\end{align}
but we cannot further absorb the higher terms because they have a real, meaningful momentum dependence.
So, what could we even mean by something like `unitary gas' in 2D?
It is not possible, for example, to have a momentum-independent phase shift.
The issue is that even if $\sigma_2$ and all higher shape parameters vanish, the $\log(pa)$ vanishes when its argument is $1$ and any momentum is allowed.
So if $a$ is fixed by the Hamiltonian then there is unavoidable momentum dependence.
Note that we are guaranteed that $a>0$.
The coefficient of the $p^2$ term has units of length${}^2$; the effective range $r_e$ has dimensions of length and is given by
\begin{align}
	r_e^2 = \sigma_2;
\end{align}
its square may be either sign.
In this form \eqref{ere} the effective range expansion matches the convention of \Ref{Beane:2022wcn}.
We will call this expression \eqref{ere} `the' effective range expansion.

\subsection{Differing Conventions}

Is $a$ what is meant by the 2D scattering length?  Or something else?
One other way to understand scattering length is to adopt the \emph{geometric meaning} of the scattering length: where, for a hard-disk potential, the wavefunction hits 0 after linear extrapolation  with linear extrapolation from the disk's boundary.
According to \Refs{Adhikari:1986a,Adhikari:1986b,Khuri:2008ib,Galea:2017jhe}, we recover this geometric meaning when we pick
\begin{align}
	\cot \delta_0(p) = \frac{2}{\pi}\left[\log \frac{pA}{2} + \gamma\right] + \oneover{4} R_e^2 p^2 + \cdots
	\label{eq:ere in the hard-disk convention}
\end{align}
where $\gamma = 0.577\ldots$ is the Euler–Mascheroni constant and $R_e$ has a similar geometric meaning (see \Ref{Galea:2017jhe} (18)).
We can match this form to the form \eqref{ere} where we totally absorb the constant into $a$ if we pick
\begin{align}
	a &= \frac{e^{\gamma}}{2} A = (0.8905362090\cdots) A.
	&
	\sigma_2 = r_e^2 &= \oneover{4} R_e^2
\end{align}

\emph{At the cold-atoms/nuclear matter ECT* program we saw experimental results always described as a function of $\log k_F a$ without specifying which convention was meant.}
