\section{Effective Range Expansion 2D}

To understand what we mean by this theory we'll need to take some sort of continuum / infinite volume limit (since our Hamiltonian has no length scale in it, this should be reached by taking $N_x$ large).
As in 3D this means tuning some scattering amplitude to match the target scattering data.

For partial wave labeled by angular momentum $\ell$ in 2D the effective range expansion reads
\begin{align}
	\cot \delta_\ell(p)
	&=
			\frac{2}{\pi} \log(p R_\ell)
		-	\frac{1}{a_\ell} p^{-2\ell}
		+	\frac{1}{2} r_\ell p^{2-2\ell}
		+	\cdots.
	\\
	\cot \delta_0(p)
	&=
			\frac{2}{\pi} \log(p R_0)
		-	\frac{1}{a_0}
		+	\frac{1}{2} r_0 p^2
\end{align}
where $R$ is an arbitrary dimensionful scale and (in the second line) the `scattering length' $a_0$ is actually dimensionless (note the units on the LHS) and $r_0$ has dimension +2.
All even-spatial-dimension effective range expansions get this log, and every non-log term gets an additional factor of $p^{2-D}$ which happens to vanish when $D=2$.
As a result, what's strange/magical about $D=2$ is that we can absorb $a_0$ into the first time by changing the scale in the log,
\begin{align}
	A &= R_0 e^{-\pi/2a_0}
	&
	\cot \delta(p)
	&=
			\frac{2}{\pi} \log(p A)
		+	\frac{1}{2} r_0 p^2
		+	\cdots
	\label{eq:no-constant}
\end{align}
but we cannot further absorb the higher terms because they have a real, meaningful momentum dependence.
So, could we even mean by something like `unitary gas' in 2D?
It is not possible, for example, to have a momentum-independent phase shift.
The issue is that even if $r$ and all higher shape parameters vanish, the $\log(pA)$ vanishes when its argument is $1$ and any momentum is allowed.
So if $A$ is fixed by the Hamiltonian then there is unavoidable momentum dependence.

Is $A$ what is meant by the 2D scattering length?  Or something else?
At the cold-atoms/nuclear matter ECT* program we saw experimental results always described as a function of $\log k_F a$.
Let's not forget that there is a \emph{geometric meaning} to the scattering length: where, with linear extrapolation, the wavefunction hits $0$.
According to \Refs{Adhikari:1986a,Adhikari:1986b,Khuri:2008ib,Galea:2017jhe}, we recover this geometric meaning when we pick
\begin{align}
	\cot \delta(p) = \frac{2}{\pi}\left[\log \frac{pa}{2} + \gamma\right] + \oneover{4} r_e^2 p^2 + \cdots
	\label{eq:ere}
\end{align}
where $\gamma = 0.577\ldots$ is the Euler–Mascheroni constant and $r_e$ has a similar geometric meaning (see \Ref{Galea:2017jhe} (18)).
We will call this expression \eqref{ere} `the' effective range expansion.
We can match this form to the form \eqref{no-constant} where we totally absorb the constant into $A$ if we pick
\begin{equation}
	A = \frac{e^{\gamma}}{2} a.
\end{equation}
Note that we are guaranteed that $a>0$.

