\subsection{The Kinetic, Potential, and Internal Energies}\label{sec:energy}

The internal energy
\begin{align}
	\tilde{U}
	= - \partial_{\tilde{\beta}} \log Z
	=
	- \oneover{Z} \partial_{\tilde{\beta}} \tr{ e^{-\tilde{\beta}(\tilde{H}-\tilde{\mu} \tilde{N} - \tilde{h}\cdot \tilde{S})} }
	=
	\oneover{Z} \tr{ e^{-\tilde{\beta}(\tilde{H}-\tilde{\mu} \tilde{N} - \tilde{h}\cdot \tilde{S})} (\tilde{H}-\tilde{\mu} \tilde{N} - \tilde{h}\cdot \tilde{S})}
	= + \expectation{\tilde{H} - \tilde{\mu} \tilde{N} - \tilde{h}\cdot \tilde{S}}
\end{align}
We may also think of differentiating our computational partition function \eqref{computational partition function}, which may be conveniently accomplished by automatic differentiation.
However, knowing that the contact \Secref{contact} computed that way has much greater variance, we will compute this expectation value by converting the operator into Wick contractions.

As we already know how to compute the number \eqref{fermionic particle number} and spin, we will focus here on computing the Hamiltonian \eqref{dimensionless hamiltonian}.
First let us compute the kinetic energy,
\begin{align}
	\expectation{\tilde{K}}
	&=
	\expectation{ \sum_{ab\sigma} \tilde{\psi}\adjoint_{a\sigma} \tilde{\kappa}_{ab} \tilde{\psi}_{b\sigma} }
	=
	\expectation{ \sum_{ab\sigma} \G^{\sigma \sigma}_{ab} \tilde{\kappa}_{ab} }.
\end{align}
The potential energy includes the bilinear offset by $\tilde{C}_0$,
\begin{align}
	\expectation{\tilde{V}}
	&=
	\expectation{ \half \sum_{ab} \tilde{n}_a \tilde{V}_{ab} \tilde{n}_b - \frac{\Volume \tilde{C}_0}{2}\sum_{a} n_a }
	\nonumber\\
	\expectation{\tilde{V}}
	+ \frac{\Volume \tilde{C}_0}{2} \expectation{\tilde{N}}
	&=
	\expectation{ \half \sum_{ab} \tilde{n}_a \tilde{V}_{ab} \tilde{n}_b }
\end{align}
where $\tilde{V}_{ab}$ is a sum of Lego spheres \eqref{Lego sphere} and the $\expectation{\tilde{N}=N}$ is known.
The potential is translationally invariant $\tilde{V}_{a,a+r} = \tilde{V}_{0,r}$ and we can shift the indices by equal amounts to find $\tilde{V}_{a,a+r}=\tilde{V}_{a-r,a}$.
Let us now rephrase the sum over the first coordinate as a sum over a relative coordinate,
\begin{align}
	\expectation{\tilde{V}}
	+ \frac{\Volume \tilde{C}_0}{2} \expectation{\tilde{N}}
	&=
	\expectation{ \half \sum_{rb} \tilde{n}_{b-r} \tilde{V}_{b-r,b} \tilde{n}_b }
	\nonumber\\
	&=
	\expectation{ \half \sum_{rb} \tilde{n}_{b-r} \tilde{V}_{0,r} \tilde{n}_b }
	\nonumber\\
	&=
	\expectation{ \half \sum_{r} \tilde{V}_{0,r} \sum_b \tilde{n}_b \tilde{n}_{b-r} }
	\nonumber\\
	&=
	\expectation{ \frac{\Volume}{2} \sum_{r} \tilde{V}_{0,r} \oneover{\Volume} \sum_b \tilde{n}_b \tilde{n}_{b-r} }
	\nonumber\\
	&=
	\frac{\Volume}{2} \sum_{r} \tilde{V}_{0,r} \expectation{ (\tilde{n}*\tilde{n})_r }
\end{align}
where we can use the convolved density-density correlator \eqref{nn}.
So, the potential energy is
\begin{align}
	\expectation{\tilde{V}}
	&=
	\frac{\Volume}{2} \sum_{r} \tilde{V}_{0,r} \expectation{ (\tilde{n}*\tilde{n})_r }
	- \frac{\Volume \tilde{C}_0}{2} \expectation{\tilde{N}}.
\end{align}
We can now evaluate $\tilde{U}$!

Since the physical energies have dimension $[(ML^2)\inverse]$, the dimensionless energies are the physical ones times $ML^2$.
Therefore, for example, $\tilde{U} = U ML^2$ is dimensionless but doubly-extensive.
To get an intensive quantity we can normalize by $k_F^4$, as is done for the contact to produce a density \eqref{contact density},
\begin{align}
	\frac{u}{k_F^4} = \frac{MU}{k_F^4 L^2} = \frac{UML^2}{(k_F L)^4} = \frac{UML^2}{(2\pi N)^2} = \frac{\tilde{U}}{(2\pi\tilde{N})^2}
	\label{eq:internal energy density}
\end{align}
where $u=MU/L^2$.  (The contact \eqref{contact} is also defined with an $M$ upstairs.)
