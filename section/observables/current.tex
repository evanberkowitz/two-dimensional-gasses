\subsection{Current}\label{sec:current}

Currents describe how quantities flow from place to place.
When the quantity $q$ is conserved it obeys the \emph{continuity equation},
\begin{align}
	\grad \cdot \vec{j} = - \partial_t q.
	\label{eq:continuity}
\end{align}
Integrating over the whole space yields the global current on the left and the temporal change of the global $Q$, which is zero.
Therefore we expect the spatial sum of the divergence of local currents to be 0.

To find the lattice-native current that satisfies the continuity equation \eqref{continuity} we compute the time derivative of $q$ as a commutator,
\begin{align}
	ML^2\grad\cdot\vec{j}_z = -ML^2 \partial_t q_z = -i[ML^2H, q_z].
\end{align}

Let us first consider the flow of particle number.
Because the interaction and $g$-dependent bilinear terms depend only on $n$s, which commute, we find
\begin{align}
	\grad\cdot\tilde{j}_z = -ML^2 \partial_t \tilde{n}_x = -i[\tilde{H}, \tilde{n}_z]
	&=
		-i \left[
			\sum_{xy} \tilde{\psi}\adjoint_x \tilde{\kappa}_{xy} \tilde{\psi}_y,
		\tilde{n}_z\right]
	\nonumber\\
	&=
		i \sum_{xy} \tilde{\kappa}_{xy} \left[ \tilde{n}_z, \tilde{\psi}\adjoint_x \tilde{\psi}_y\right]
	\nonumber\\
	&=
		i \sum_{xy} \tilde{\kappa}_{xy} \left(
			\left[ \tilde{n}_z, \tilde{\psi}\adjoint_x \right]\tilde{\psi}_y
			+
			\tilde{\psi}\adjoint_x \left[ \tilde{n}_z, \tilde{\psi}_y\right]
		\right)
	\nonumber\\
	&=
		i \sum_{xy} \tilde{\kappa}_{xy} \left(
			+ \delta_{zx} \tilde{\psi}\adjoint_x \tilde{\psi}_y
			-
			\tilde{\psi}\adjoint_x \delta_{zy} \tilde{\psi}_y
		\right)
	\nonumber\\
	&=
		i \sum_{y} \left(
			\tilde{\psi}\adjoint_z \tilde{\kappa}_{zy} \tilde{\psi}_y
		-	\tilde{\psi}\adjoint_y \tilde{\kappa}_{yz} \tilde{\psi}_z
		\right)
\end{align}
Expressing $j$ in terms of its Fourier modes,
\begin{align}
	\grad\cdot \tilde{j}_z
	=
	\grad \cdot \oneover{\Volume}\sum_k \tilde{j}_k e^{+2\pi i k z/N_x}
	=
	\oneover{\Volume}\sum_k \tilde{j}_k \cdot \frac{2\pi i k}{N_x} e^{+2\pi i k z/N_x}
	=
	\oneover{\Volume}\sum_k \left( \frac{2\pi i k}{N_x} \cdot \tilde{j}_k \right) e^{+2\pi i k z/N_x}.
	\label{eq:divergence of j}
\end{align}
Substituting the dimensionless kinetic matrix \eqref{dimensionless hamiltonian} one finds
\begin{align}
	i \sum_{y} \left(
		\tilde{\psi}\adjoint_z \tilde{\kappa}_{zy} \tilde{\psi}_y
	-	\tilde{\psi}\adjoint_y \tilde{\kappa}_{yz} \tilde{\psi}_z
	\right)
	&=
	i \sum_{y} \left(
		\tilde{\psi}\adjoint_z \sum_{k} \frac{(2\pi k)^2}{2 \Volume} e^{-2\pi i k \cdot(z-y)/N_x} \tilde{\psi}_y
	-	\tilde{\psi}\adjoint_y \sum_{k} \frac{(2\pi k)^2}{2 \Volume} e^{-2\pi i k \cdot(y-z)/N_x} \tilde{\psi}_z
	\right)
	\nonumber\\
	&=
	i \sum_{y} \left(
		\tilde{\psi}\adjoint_z \sum_{k} \frac{(2\pi k)^2}{2 \Volume} e^{+2\pi i k \cdot(z-y)/N_x} \tilde{\psi}_y
	-	\tilde{\psi}\adjoint_y \sum_{k} \frac{(2\pi k)^2}{2 \Volume} e^{-2\pi i k \cdot(y-z)/N_x} \tilde{\psi}_z
	\right)
	\nonumber\\
	&=
	i\sum_k e^{+2\pi i k z / N_x}\left(
		\frac{(2\pi k)^2}{2\Volume} \sum_y e^{-2\pi i k \cdot y / N_x} i \left(\tilde{\psi}\adjoint_z \tilde{\psi}_y  - \tilde{\psi}\adjoint_y \tilde{\psi}_z \right)
	\right)
	\label{eq:current operator intermediate}
\end{align}
where after the first line we send $k\goesto-k$ in the first term.
Identifying these two results \eqref{divergence of j} and \eqref{current operator intermediate} we find
\begin{align}
	\sum_{z} e^{-2\pi i p z / N_x} \oneover{V} \sum_k \left(\frac{2\pi i k}{N_x} \cdot \tilde{j}_k \right) e^{+2\pi i k z / N_x}
	&=
	\sum_{z} e^{-2\pi i p z / N_x} i\sum_k e^{+2\pi i k z / N_x} \left(
		\half \left(\frac{2\pi k}{N_x}\right)^2 \sum_y e^{-2\pi i k \cdot y / N_x} i \left(\tilde{\psi}\adjoint_z \tilde{\psi}_y  - \tilde{\psi}\adjoint_y \tilde{\psi}_z \right)
	\right)
	\nonumber\\
	\oneover{\Volume} \sum_k \left( \frac{2\pi i k}{N_x} \cdot \tilde{j}_k \right) \Volume \delta_{k,p}
	&=
	\frac{i}{2}\sum_k
		\left(\frac{2\pi k}{N_x}\right)^2 \sum_{yz} e^{-2\pi i [k \cdot (y-z) + p \cdot z] / N_x} \left(\tilde{\psi}\adjoint_z \tilde{\psi}_y  - \tilde{\psi}\adjoint_y \tilde{\psi}_z \right)
	\nonumber\\
	\frac{2\pi i p}{N_x} \cdot \tilde{j}_p
	&=
	\frac{i}{2} \sum_k
		\left(\frac{2\pi k}{N_x}\right)^2 \sum_{yz} e^{-2\pi i [k \cdot y - (k-p) \cdot z] / N_x} \left(\tilde{\psi}\adjoint_z \tilde{\psi}_y  - \tilde{\psi}\adjoint_y \tilde{\psi}_z \right)
	\nonumber\\
	\frac{2\pi p}{N_x} \cdot \tilde{j}_p
	&=
	\frac{1}{2} \sum_k \left(\frac{2\pi k}{N_x}\right)^2 \left( \tilde{\psi}\adjoint_{k-p} \tilde{\psi}_k - \tilde{\psi}\adjoint_{-k} \tilde{\psi}_{p-k} \right)
	\nonumber\\
	&=
	\frac{1}{2} \sum_k \left(\frac{2\pi k}{N_x}\right)^2 \left( \tilde{\psi}\adjoint_{k-p} \tilde{\psi}_k - \tilde{\psi}\adjoint_{k} \tilde{\psi}_{p+k} \right)
\end{align}
where, in the last step, we again send $k\goesto-k$.
Now we shift $k$ (mod the Brillouin zone) in the second term to match the first,
\begin{align}
	\frac{2\pi p}{N_x} \cdot \tilde{j}_p
	&=
	\frac{1}{2} \sum_k \left(\frac{2\pi k}{N_x}\right)^2 \tilde{\psi}\adjoint_{k-p} \tilde{\psi}_k - \left(\frac{2\pi k}{N_x}\right)^2 \tilde{\psi}\adjoint_{k} \tilde{\psi}_{p+k}
	\nonumber\\
	&=
	\frac{1}{2} \sum_k \left(\frac{2\pi k}{N_x}\right)^2 \tilde{\psi}\adjoint_{k-p} \tilde{\psi}_k - \left(\frac{2\pi (k-p)}{N_x}\right)^2 \tilde{\psi}\adjoint_{k-p} \tilde{\psi}_{k}
	\nonumber\\
	&=
	\frac{1}{2} \sum_k \left[\left(\frac{2\pi k}{N_x}\right)^2 - \left(\frac{2\pi (k-p)}{N_x}\right)^2 \right] \tilde{\psi}\adjoint_{k-p} \tilde{\psi}_{k}
	\nonumber\\
	&=
	\frac{1}{2} \sum_k \left(\frac{2\pi}{N_x}\right)^2\left[ 2pk - p^2 \right] \tilde{\psi}\adjoint_{k-p} \tilde{\psi}_{k}
	\nonumber\\
	&=
	\frac{2\pi p}{N_x} \cdot \left(\frac{1}{2} \sum_k \left(\frac{2\pi}{N_x}\right)\left[ 2k - p \right] \tilde{\psi}\adjoint_{k-p} \tilde{\psi}_{k} \right)
	\nonumber\\
	\tilde{j}_p
	&=
	\frac{1}{2} \sum_k \left(\frac{2\pi [ 2k-p ]}{N_x}\right) \tilde{\psi}\adjoint_{k-p} \tilde{\psi}_{k}
	=
	\frac{1}{2} \sum_k \left(\frac{2\pi [ 2k+p ]}{N_x}\right) \tilde{\psi}\adjoint_{k} \tilde{\psi}_{k+p}
	\label{eq:current in momentum space}
\end{align}
to find the current in momentum space.

We can now Fourier transform back to position space and express the momentum-basis ladder operators in position space
\begin{align}
	\tilde{j}_x
	=
	\oneover{\Volume} \sum_p e^{+2\pi i p \cdot x / N_x} \tilde{j}_p 
	&=
	\oneover{\Volume} \sum_p e^{+2\pi i p \cdot x / N_x} 
	\frac{1}{2} \sum_k \left(\frac{2\pi [ 2k+p ]}{N_x}\right) \tilde{\psi}\adjoint_{k} \tilde{\psi}_{k+p}
	\nonumber\\
	&=
	\oneover{2\Volume} \sum_p e^{+2\pi i p \cdot x / N_x} 
	\sum_k \left(\frac{2\pi [ 2k+p ]}{N_x}\right) \sum_{yz} \tilde{\psi}\adjoint_{y} e^{2\pi i (ky - (k+p)z) / N_x} \tilde{\psi}_{z}
	\nonumber\\
	&=
	\oneover{2\Volume}
	\sum_{yz} \tilde{\psi}\adjoint_{y}  \tilde{\psi}_{z}
	\sum_{pk} e^{+2\pi i [ k (y-z) + p  (x-z) ]  / N_x} \left(\frac{2\pi [ 2k+p ]}{N_x}\right)
\end{align}
Since $p$ is summed we can shift $q=p+k$ to get
\begin{align}
	\tilde{j}_x
	&=
	\oneover{2\Volume}
	\sum_{yz} \tilde{\psi}\adjoint_{y}  \tilde{\psi}_{z}
	\sum_{qk} e^{+2\pi i [ k (y-z) + (q-k)  (x-z) ]  / N_x} \left(\frac{2\pi [ k+q ]}{N_x}\right)
	\nonumber\\
	&=
	\oneover{2\Volume}
	\sum_{qk} e^{+2\pi i (q-k) x / N_x} \left(\frac{2\pi [ k+q ]}{N_x}\right)
	\sum_{yz} e^{+2\pi i ky / N_x} \tilde{\psi}\adjoint_{y} \tilde{\psi}_{z} e^{-2\pi i q z / N_x}
	\\
	&=
	\oneover{2\Volume}
	\sum_{qk} e^{+2\pi i (q-k) x / N_x} \left(\frac{2\pi [ k+q ]}{N_x}\right)
	\tilde{\psi}\adjoint_{k} \tilde{\psi}_{q}
	\label{eq:current hermitian}
\end{align}
from which it is straightforward to show $\tilde{j}\adjoint_x = \tilde{j}_x$ so that $\tilde{j}_x$ is observable.

We can now Wick contract using the propagator in momentum space \eqref{propagator in momentum space}
\begin{align}
	\expectation{\tilde{j}_x}
	&=
	\expectation{
		\oneover{2} \sum_{pkq} e^{+2\pi i (q-k)x / N_x} \left(\frac{2\pi [ k+q ]}{N_x}\right)
		\G_{kq}
		}
\end{align}
where we sum over spins.
A reasonable algorithm to compute this is to construct a tensor $T_{kq}^i = 2\pi(k+q)^i/N_x$ and do element-wise multiplication,
\begin{align}
	\expectation{\tilde{j}_x}
	=
	\expectation{
		\frac{1}{2} \sum_{kq} e^{2\pi i(q-k) x / N_x} \left[ T^i \G\right]_{kq}
	}
	&=
	\expectation{
		\frac{\Volume}{2\Volume} \sum_{yzkq} \delta_{xy}\delta_{xz} e^{2\pi i(qz-ky) / N_x} \left[ T^i \G\right]_{kq}
	}
	\nonumber\\
	&=
	\expectation{
		\frac{\Volume}{2} \sum_{yz} \delta_{xy} \delta_{xz} \oneover{\Volume}\sum_{kq} e^{2\pi i(qz-ky) / N_x} \left[ T^i \G\right]_{kq}
	}
\end{align}
which are the diagonal elements of a double Fourier transform; we can also express $\tilde{j}_x$ as a trace with a projector.
Because of the Fourier transforms we can compute $\expectation{\tilde{j}_x}$ at $\order{\Volume^2 \log^2 \Volume}$ cost.

Of course, by rotational symmetry, the average $\expectation{\tilde{j}}=0$ and so we are already guaranteed that the continuity equation \eqref{continuity} is satisfied on average.
Let us check that this $\tilde{j}$ satisfies the continuity equation \eqref{continuity} measurement-by-measurement.
We take the divergence of $\tilde{j}$,
\begin{align}
	-\partial_t N = \sum_x (\grad\cdot \tilde{j})_x
	&=
	\sum_x \oneover{2\Volume} \sum_{kq} e^{+2\pi i (q-k) x / N_x} \left(\frac{2\pi i(q-k)}{N_x}\right) \cdot \left(\frac{2\pi [ k+q ]}{N_x}\right)
		\G_{kq}
	\nonumber\\
	&=
	\frac{4\pi^2i}{2\Volume^2} \sum_{kq}  (q^2-k^2)
		\G_{kq} \sum_x e^{+2\pi i (q-k) x / N_x}
	\nonumber\\
	&=
	\frac{4\pi^2i}{2\Volume^2} \sum_{kq}  (q^2-k^2)
		\G_{kq} \Volume \delta_{kq} = 0
\end{align}
and so this current indeed corresponds to the lattice-exact conservation of $N$, which is true as an operator and therefore configuration-by-configuration.

Computing the spin current proceeds similarly, but the propagator gets a $\half \sigma$, and we also should add a contribution from the commutator of $\tilde{\vec{h}}\cdot\tilde{\vec{S}}$ with the local $s$.

We can compute a current-current two-point function or the local square of the current.
Let's start with
\begin{align}
	\tilde{j}\adjoint_x \cdot \tilde{j}_y
	&=
	\oneover{4\Volume^2}\sum_{q'k'kq} e^{2\pi i[ -(q'-k')x + (q-k) y]/N_x} \left(\frac{2\pi}{N_x}\right)^2 (k'+q')\cdot(k+q)
	\tilde{\psi}\adjoint_{q'\sigma} \tilde{\psi}_{k'\sigma} \tilde{\psi}\adjoint_{k\tau} \tilde{\psi}_{q\tau}
\end{align}
which we may then simplify by setting $y=x$ or convolving $y$ with $x$.
Putting into normal order,
\begin{align}
	\tilde{j}\adjoint_x \cdot \tilde{j}_y
	&=
	\oneover{4}\sum_{q'k'kq} e^{2\pi i[ -(q'-k')x + (q-k) y]/N_x} \left(\frac{2\pi}{N_x}\right)^2 (k'+q')\cdot(k+q)
	\left(\G^{\tau\tau}_{kq}\G^{\sigma\sigma}_{q'k'} - \G^{\tau\sigma}_{kk'} \G^{\sigma\tau}_{q'q} + \delta^{\tau\sigma}_{kk'} \G^{\sigma\tau}_{q'q}\right)
\end{align}
The naive algorithm for computing this would construct two tensors that require $\order{\Volume^4}$ memory (the dot product and the contractions), multiply them element-wise, and fourier transform.

We can avoid the $\order{\Volume^4}$ footprint, however, if we are a bit clever.
Let us construct three additional quantities,
\begin{align}
	[\p\G]^{\sigma\tau}_{kiq} &= \frac{2\pi}{N_x} k^i \G^{\sigma\tau}_{kq}
	&
	[\G \p]^{\sigma\tau}_{kqi} &= \frac{2\pi}{N_x} \G^{\sigma\tau}_{kq} q^i
	&
	[\p\G \p]^{\sigma\tau}_{kiqj} &= \left(\frac{2\pi}{N_x}\right)^2 k^i \G^{\sigma\tau}_{kq} q^j,
\end{align}
all of $\order{\Volume^2}$ memory and taking $\order{\Volume^{\text{3 or 4}}}$ to construct.
\begin{align}
	\tilde{j}\adjoint_x \cdot \tilde{j}_y
	=&
	\oneover{4}\sum_{q'k'kq} e^{2\pi i[ -(q'-k')x + (q-k) y]/N_x}
	\nonumber\\
	\Bigg\{&
		[\p\G]^{\tau\tau}_{kiq} [\G\p]^{\sigma\sigma}_{q'k'i}
	+	[\p\G]^{\tau\tau}_{kiq} [\p\G]^{\sigma\sigma}_{q'ik'}
	+	[\G\p]^{\tau\tau}_{kqi} [\G\p]^{\sigma\sigma}_{q'k'i}
	+	[\G\p]^{\tau\tau}_{kqi} [\p\G]^{\sigma\sigma}_{q'ik'}
	\nonumber\\
	&
	-	[\p\G\p]^{\tau\sigma}_{kik'i} \G^{\sigma\tau}_{q'q}
	-	[\G\p]^{\tau\sigma}_{kk'i} [\G\p]^{\sigma\tau}_{q'qi}
	-	[\p\G]^{\tau\sigma}_{kik'} [\p\G]^{\sigma\tau}_{q'iq}
	-	\G^{\tau\sigma}_{kk'} [\p\G\p]^{\sigma\tau}_{q'iqi}
	\nonumber\\
	&
	+\left(
		[\p\delta\p]^{\tau\sigma}_{kik'i} \G^{\sigma\tau}_{q'q}
	+	[\delta\p]^{\tau\sigma}_{kk'i} [\G\p]^{\sigma\tau}_{q'qi}
	+	[\p\delta]^{\tau\sigma}_{kik'} [\p\G]^{\sigma\tau}_{q'iq}
	+	\delta^{\tau\sigma}_{kk'} [\p\G]^{\sigma\tau}_{q'iqi}
	\right)
	\Bigg\}
\end{align}
where we defined the momentum-multiplied $[\p\delta]$, $[\delta\p]$, and $[\p\delta\p]$ by analogy.
We can now apply the fourier transformation term-by-term and indeed factor-by-factor, as long as we carefully pay attention to which resulting FTs are in $x$ and which are in $y$.
Finally, we can do the sums on $i$, $\sigma$, and $\tau$.
This is a substantially smarter algorithm, though it's much more annoying to code correctly.
