\subsection{Pairing}\label{sec:pairing}

\Refs{PhysRevA.92.033603,PhysRevLett.123.136402,PhysRevLett.129.076403} give the zero-momentum (back-to-back) singlet pairing matrix
\begin{align}
    M_{kk'} &=
    \expectation{\Delta\adjoint_k \Delta_{k'}}
    -   \delta_{kk'}
        \expectation{\tilde{\psi}\adjoint_{k\up} \tilde{\psi}_{k\up}}
        \expectation{\tilde{\psi}\adjoint_{-k\dn} \tilde{\psi}_{-k\dn}}
    &
    \Delta\adjoint_k &= \tilde{\psi}\adjoint_{k\up} \tilde{\psi}\adjoint_{-k\dn}
    \label{eq:PRL129076403 pairing}
\end{align}
where the $\Delta$ operator is like a Cooper pair.
In the free case
\begin{align}
    \expectation{\Delta\adjoint_k \Delta_{k'}}
    =
    \Volume^2\expectation{ \G^{\up\up}_{kk'} \G^{\dn\dn}_{-k',-k} }
    =
    \Volume^2\expectation{ \G^{\up\up}_{kk'} } \expectation{ \G^{\dn\dn}_{-k',-k} }
    =
    \delta_{kk'}
    \expectation{\tilde{\psi}\adjoint_{k\up} \tilde{\psi}_{k\up}}
    \expectation{\tilde{\psi}\adjoint_{-k\dn} \tilde{\psi}_{-k\dn}}
\end{align}
because the different spins' Hilbert spaces separate.
This is subtracted so that everything in $M$ arises from interactions.

They say that the \emph{condensate fraction} $n_c$ is the `leading eigenvalue' of $M$ normalized by $N/2$; presumably the smallest eigenvalue.
They identify the BKT transition temperature by finding the peak of
\begin{align}
    \frac{d n_c}{d(T/T_F)}
\end{align}
as a function of $T/T_F$ and extrapolating to the thermodynamic limit.

We define four pairing operators
\begin{align}
    \tilde{\Delta}\adjoint_{S,m,k} &= \tilde{\psi}\adjoint_{\alpha,k} \left(P\adjoint_{S,m}\right)_{\alpha\beta} \tilde{\psi}\adjoint_{\beta,-k}
    &
    P_{0,0} &= \frac{-i \sigma_2}{\sqrt{2}}
    &
    P_{1,0} &= \frac{\sigma_1}{\sqrt{2}}
    &
    P_{1,\pm1} &= \oneover{2}\left( \one \pm \sigma_3 \right)
\end{align}
which have definite quantum numbers total spin $S$ and $S_z=m$.
The pairing \eqref{PRL129076403 pairing} in \Refs{PhysRevA.92.033603,PhysRevLett.123.136402,PhysRevLett.129.076403} uses $P = (P_{0,0} + P_{1,0})/\sqrt{2}$.
Just as the noninteracting piece is subtracted above \eqref{PRL129076403 pairing}, we do the same, defining
\begin{align}
    M^{S,m}_{k,k'} &= \expectation{ \tilde{\Delta}\adjoint_{S,m,k} \tilde{\Delta}_{S,m,k'} - \delta_{kk'} \tilde{\Delta}\adjoint_{S,m,k} \tilde{\Delta}_{S,m,k}}
\end{align}
According to the \R{leggett:2006}, a single extensive eigenvalue of this matrix corresponds indicates long-range order and according to \R{PhysRevLett.129.076403} the corresponding eigenvector is `the pairing wavefunction' in reciprocal space.
Also according to \R{leggett:2006} more than one extensive eigenvalue indicates a `fractured condensate'.

The Wick contractions of the more generic
\begin{align}
    \expectation{ \tilde{\Delta}\adjoint_{S,m,k} \tilde{\Delta}_{S,m,q} }
    =&
    \left(P\adjoint_{S,m}\right)_{\alpha\beta} \left(P^{\phantom{\dagger}}_{S,m}\right)_{\sigma\tau} \expectation{ 
        \tilde{\psi}\adjoint_{\alpha,k} \tilde{\psi}\adjoint_{\beta,-k}
        \tilde{\psi}_{\sigma,q} \tilde{\psi}_{\tau,q} }
    \nonumber\\
    =&
    \left(P\adjoint_{S,m}\right)_{\alpha\beta} \left(P^{\phantom{\dagger}}_{S,m}\right)_{\sigma\tau}
    \sum_{xyzw} e^{+i(kx-ky+qz-qw)}
    \expectation{ 
        \tilde{\psi}\adjoint_{\alpha,x} \tilde{\psi}\adjoint_{\beta,y}
        \tilde{\psi}_{\sigma,z} \tilde{\psi}_{\tau,w}
        }
    \nonumber\\
    =&
    \left(P\adjoint_{S,m}\right)_{\alpha\beta} \left(P^{\phantom{\dagger}}_{S,m}\right)_{\sigma\tau}
    \sum_{xyzw} e^{+i(kx-ky+qz-qw)}
    \expectation{ 
            \G^{\alpha\tau}_{xw} \G^{\beta\sigma}_{yz}
        -   \G^{\alpha\sigma}_{xz} \G^{\beta\tau}_{yw}
        }
    \nonumber\\
    =&
    \left(P\adjoint_{S,m}\right)_{\alpha\beta} \left(P^{\phantom{\dagger}}_{S,m}\right)_{\sigma\tau}
    N_x^4 \times
    \\
    &
    \expectation{ 
            (\texttt{ifft}_{x \goesto k} \texttt{fft}_{w \goesto q}\G^{\alpha\tau}_{xw})
            (\texttt{fft}_{y \goesto k} \texttt{ifft}_{z \goesto q}\G^{\beta\sigma}_{yz})
        -   (\texttt{ifft}_{x \goesto k} \texttt{ifft}_{z \goesto q}\G^{\alpha\sigma}_{xz})
            (\texttt{fft}_{y \goesto k} \texttt{fft}_{w \goesto q} \G^{\beta\tau}_{yw})
        }
    \nonumber
\end{align}
using the Fourier transforms defined in \Tabref{ft} and the propagator contraction \eqref{propagator}.
