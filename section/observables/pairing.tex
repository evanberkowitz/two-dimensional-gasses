\subsection{Pairing}\label{sec:pairing}

\Refs{PhysRevA.92.033603,PhysRevLett.123.136402,PhysRevLett.129.076403} give the zero-momentum (back-to-back) singlet pairing matrix
\begin{align}
	M_{kk'} &=
	\expectation{\Delta\adjoint_k \Delta_{k'}}
	-	\delta_{kk'}
		\expectation{\tilde{\psi}\adjoint_{k\up} \tilde{\psi}_{k\up}}
		\expectation{\tilde{\psi}\adjoint_{-k\dn} \tilde{\psi}_{-k\dn}}
	&
	\Delta\adjoint_k &= \tilde{\psi}\adjoint_{k\up} \tilde{\psi}\adjoint_{-k\dn}
\end{align}
where the $\Delta$ operator is like a Cooper pair.
They say that the \emph{condensate fraction} $n_c$ is the `leading eigenvalue' of $M$ normalized by $N/2$; presumably the smallest eigenvalue.
They identify the BKT transition temperature by finding the peak of
\begin{align}
	\frac{d n_c}{d(T/T_F)}
\end{align}
as a function of $T/T_F$ and extrapolating to the thermodynamic limit.
