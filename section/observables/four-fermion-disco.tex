\subsection{The Four-Fermion Disconnected Piece}

We can think of four-body operators that are not derivatives of the partition function with respect to $\tilde{\mu}$ or $\tilde{h}$ as derivatives of the partition function nevertheless, which may be useful in understanding thermodynamic properties.
We can focus on a generic four-quark operator,
\begin{align}
	\O = \tilde{\psi}\adjoint_{q'\sigma'} \tilde{\psi}_{k'\tau'} \tilde{\psi}\adjoint_{k\tau} \tilde{\psi}_{q\sigma}
\end{align}
with four generic momenta and spins.
We can think of many different observables as being matrix elements of this kind, with subsequent manipulation and tying together of the eight free indices.

We have to use the fermionic derivatives \eqref{fermionic derivatives} of the fermionic generating function \eqref{fermionic generating function} to find the (quantum-average-)disconnected pieces that go along with our operator.
Moving back to position space for simplicity,
\begin{align}
	\O &= \sum_{y'x'xy} e^{2\pi i [q'y' - k'x' + kx - qy]/N_x} \O'
	&
	\O' &= \tilde{\psi}\adjoint_{y'\sigma'} \tilde{\psi}_{x'\tau'} \tilde{\psi}\adjoint_{x\tau} \tilde{\psi}_{y\sigma}
\end{align}
So, we figure out the derivatives required in position space and Fourier transform to get the required momentum operators.

Let us differentiate
\begin{align}
	\partial\adjoint_{x'\tau'} \partial_{y'\sigma'} \left[\log Z\right] \lpartial\adjoint_{y\sigma} \lpartial_{x\tau}
	=&
	\partial\adjoint_{x'\tau'} \left[\left(\partial_{y'\sigma'} Z\right) \oneover{Z} \right] \lpartial\adjoint_{y\sigma} \lpartial_{x\tau}
	\nonumber\\
	=&
	\partial\adjoint_{x'\tau'} \left[
		\left(\partial_{y'\sigma'} Z \lpartial\adjoint_{y\sigma}\right)\oneover{Z}
	-	\left( \partial_{y'\sigma'} Z\right) \oneover{Z^2} \left( Z \lpartial\adjoint_{y\sigma} \right)
	\right] \lpartial_{x\tau}
	\nonumber\\
	=&
	\Bigg[
	-	\left(\partial\adjoint_{x'\tau'} Z \right) \oneover{Z^2} \left(\partial_{y'\sigma'} Z \lpartial\adjoint_{y\sigma}\right)
	+	\oneover{Z} \left(\partial\adjoint_{x'\tau'} \partial_{y'\sigma'} Z \lpartial\adjoint_{y\sigma}\right)
	+	\frac{2}{Z^3} \left( \partial\adjoint_{x'\tau'} Z\right) \left(\partial_{y'\sigma'}Z \right)\left( Z \lpartial\adjoint_{y\sigma} \right)
	\nonumber\\
	&
	-	\left(\partial\adjoint_{x'\tau'} \partial_{y'\sigma'} Z\right)\oneover{Z^2}\left(Z \lpartial\adjoint_{y\sigma}\right)
	+	\left( \partial_{y'\sigma'} Z\right) \oneover{Z^2} \left( \partial\adjoint_{x'\tau'} Z \lpartial\adjoint_{y\sigma} \right)
	\Bigg] \lpartial_{x\tau}
\end{align}
where we carefully preserve the order of terms when needed and introduce $(-1)$s when pushing derivatives past one another (in the product rule).
We can now for simplicity drop terms that cannot possibly result in expectation values with equal creation and destruction operators and proceed without writing zero-expectation-value quantities,
\begin{align}
	\partial_{x'\tau'} \partial\adjoint_{y'\sigma'} \left[\log Z\right] \lpartial\adjoint_{y\sigma} \lpartial_{x\tau}
	=&
	\Bigg[
	-	\left(\partial\adjoint_{x'\tau'} Z \right) \oneover{Z^2} \left(\partial_{y'\sigma'} Z \lpartial\adjoint_{y\sigma}\right)
	+	\oneover{Z} \left(\partial\adjoint_{x'\tau'} \partial_{y'\sigma'} Z \lpartial\adjoint_{y\sigma}\right)
	\nonumber\\
	&
	-	\left(\partial\adjoint_{x'\tau'} \partial_{y'\sigma'} Z\right)\oneover{Z^2}\left(Z \lpartial\adjoint_{y\sigma}\right)
	\Bigg] \lpartial_{x\tau}
	\nonumber\\
	=&
	-	\left(\partial\adjoint_{x'\tau'} Z \lpartial_{x\tau}\right) \oneover{Z^2} \left(\partial_{y'\sigma'} Z \lpartial\adjoint_{y\sigma}\right)
	+	\oneover{Z} \left(\partial\adjoint_{x'\tau'} \partial_{y'\sigma'} Z \lpartial\adjoint_{y\sigma}\lpartial_{x\tau}\right)
	\nonumber\\
	&
	-	\left(\partial\adjoint_{x'\tau'} \partial_{y'\sigma'} Z\right)\oneover{Z^2}\left(Z \lpartial\adjoint_{y\sigma}\lpartial_{x\tau}\right)
\end{align}
and now we use the fermionic generating function \eqref{fermionic generating function}
\begin{align}
	\partial_{x'\tau'} \partial\adjoint_{y'\sigma'} \left[\log Z\right] \lpartial\adjoint_{y\sigma} \lpartial_{x\tau}
	=&
	+\expectation{\O' = \tilde{\psi}\adjoint_{y'\sigma'} \tilde{\psi}_{x'\tau'} \tilde{\psi}\adjoint_{x\tau} \tilde{\psi}_{y\sigma}}
	-\expectation{\tilde{\psi}_{x'\tau'} \tilde{\psi}\adjoint_{x\tau}} \expectation{\tilde{\psi}\adjoint_{y'\sigma'} \tilde{\psi}_{y\sigma}}
	-\expectation{\tilde{\psi}\adjoint_{y'\sigma'} \tilde{\psi}_{x'\tau'}} \expectation{\tilde{\psi}\adjoint_{x\tau} \tilde{\psi}_{y\sigma}}
\end{align}
so that if we are interested in an observable $\O$ with fourier transform $\O'$ the disconnected pieces are as above, fourier transformed.
