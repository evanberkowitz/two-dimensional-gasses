\subsection{The Contact}\label{sec:contact}

The contact is given by~\cite{PhysRevA.86.013626}
\begin{align}
	\hat{C} &= \frac{2 \pi M}{\hbar^2} \frac{dH}{d\log a}
	&
	\hat{C}L^2 &= 2\pi \frac{d\tilde{H}}{d\log a} 
	\label{eq:contact}
\end{align}
\Refs{Froehlich2011,PhysRevLett.109.130403} report a measurement of the contact.
Specifically they measure $C/k_F^2$ as a function of $1/\log k_F a_{2D}$ in the Fermi liquid regime.
In \R{PhysRevLett.109.130403} Fig. 4 they show a comparison with a zero-temperature fixed-node diffusion Monte Carlo calculation~\cite{PhysRevLett.106.110403}; the theory and experiment agree at $1/\log k_F a \gtrsim 0.5$ but the experiment and the theory diverge deeper into the weak-coupling regime $1/\ln k_F a \sim 0.3$.
The disagreement is only one data point, so who knows; the experimentalists did a ladder approximation and reproduced that one point.
I can't make heads or tails of the method; \R{doi:10.1063/1.443766} claims the solution is exact in the region bounded by the nodes.

We can use the chain rule to further expand the operator,
\begin{align}
    \hat{C}L^2 = 2\pi \frac{d\tilde{H}}{d C_R} \frac{d C_R}{d\log a} 
\end{align}
where the Wilson coefficients $C_R$ are those of the $A_1$ Lego sphere with radius $R$ described in \secref{Lego spheres}.
The derivative of the coefficients is accessible from the tuning to the ERE; for details see the code \sourcefile{tdg/tuning.py}.
So, the expectation value of the contact is
\begin{align}
    \expectation{\hat{C} L^2}
    = 2 \pi \frac{d C_R}{d\log a} \tr{e^{-\beta H} \left(\frac{d\tilde{H}}{d C_R} = \half \Volume \LegoSphere{R}_{ab} (\tilde{n}_a \tilde{n}_b - \tilde{n}_a \delta_{ab}) \right)}
    \label{eq:contact generic}
\end{align}
where the regrouping of the contact interaction into the bilinear term yields the last piece.
Following the same derivation as the double occupancy \eqref{double occupancy} one finds that the same-site terms conspire to yield
\begin{align}
    \expectation{\hat{C} L^2}
    &= 2 \pi \Volume \frac{d C_R}{d\log a} \LegoSphere{R}_{ab} \half\expectation{\tr{(\one + \UU)\inverse \UU \PP_a}\tr{(\one + \UU)\inverse \UU \PP_b} - \tr{(\one+\UU)\inverse \UU \PP_a (\one+\UU)\inverse \UU \PP_b}}
    \nonumber\\
    &= \pi\Volume \expectation{\sum_{Rabst} \frac{d C_R}{d\log a} \LegoSphere{R}_{ab} \left\{[(\one+\UU)\inverse \UU]^{ss}_{aa} [(\one+\UU)\inverse \UU]^{tt}_{bb} - [(\one+\UU)\inverse \UU]^{st}_{ba} [(\one+\UU)\inverse \UU]^{ts}_{ab} \right\}}
\end{align}
which, in index notation, is easily programmable.
We can also divide both sides by $\Volume$ to find
\begin{align}
    \expectation{\hat{C} \Delta x^2}
    &= \pi \expectation{\sum_{Rabst} \frac{d C_R}{d\log a} \LegoSphere{R}_{ab} \left\{[(\one+\UU)\inverse \UU]^{ss}_{aa} [(\one+\UU)\inverse \UU]^{tt}_{bb} - [(\one+\UU)\inverse \UU]^{st}_{ba} [(\one+\UU)\inverse \UU]^{ts}_{ab} \right\}}
\end{align}
In the case when only the $R=0$ sphere is included,
\begin{align}
    \expectation{\hat{C} \Delta x^2}
    &= \pi \expectation{\sum_{abst} \frac{d C_0}{d\log a} (\LegoSphere{0}_{ab} = \delta_{ab}) \left\{[(\one+\UU)\inverse \UU]^{ss}_{aa} [(\one+\UU)\inverse \UU]^{tt}_{bb} - [(\one+\UU)\inverse \UU]^{st}_{ba} [(\one+\UU)\inverse \UU]^{ts}_{ab} \right\}}
    \nonumber\\
    &= 2\pi \frac{d C_0}{d\log a} \expectation{ \DoubleOccupancy }
\end{align}
the contact is directly proportional to the double occupancy described in \secref{double occupancy}.
Of course this can be re-multiplied by $\Volume$ to find
\begin{align}
    \expectation{CL^2}
    &= 2\pi \Volume \frac{d C_0}{d\log a} \expectation{ \DoubleOccupancy }
\end{align}
in the on-site case.

The \emph{contact density} $c=C/L^2$ may be calculated to third order in $\alpha$~\cite{Beane:2022wcn}, see Fig. 12 and equation (7.6).
They normalize it by $k_F^4$,
\begin{align}
    \frac{c}{k_F^4} = \frac{(CL^2)}{(k_F L)^4} = \frac{(CL^2)}{(2\pi N)^2}
    \label{eq:contact density}
\end{align}
which is intensive, as both the numerator and denominator scale with the volume squared.
In \issue{30} we compare this quantity (at one volume and finite temperature) with the zero-temperature result of \R{Beane:2022wcn}.
