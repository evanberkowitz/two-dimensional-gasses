\subsection{Double Occupancy}\label{sec:double occupancy}

The (local) double occupancy is the expectation value of $\doubleOccupancy=\tilde{n}_\up \tilde{n}_\down$ (or the equivalent along some other direction).
The \doubleOccupancy operator is equivalent to a combination of total-number operators,
\begin{align}
    \doubleOccupancy
    &= \tilde{n}_\up \tilde{n}_\dn
    = \half \left(2 \tilde{n}_\up \tilde{n}_\dn \right)
    = \half \left( (\tilde{n}_\up + \tilde{n}_\dn)^2 - \tilde{n}_\up^2 - \tilde{n}_\dn^2\right)
    = \half \left( \tilde{n}^2 - \tilde{n} \right)
    &
    \DoubleOccupancy = \sum_a \doubleOccupancy_a
\end{align}
so that the expectation value
\begin{align}
    \expectation{2\doubleOccupancy_a}
    =
    \frac{\tr{e^{-\beta H}(\tilde{n}^2 - \tilde{n})}}{\tr{e^{-\beta H}}}
    \approx
    \frac{ \int DA\; e^{-\frac{1}{2} \sum_t A\transpose_t (-\Delta\tilde{t} \tilde{V})\inverse A_t - \frac{N_t}{2} \trlog{ -2\pi \Delta\tilde{t}\tilde{V}}}
    ~\tr{\prod_{t=1}^{N_t} e^{-\Delta \tilde{t}\tilde{K}}  e^{ \tilde{\psi}\adjoint (A_t + \Delta\tilde{t} \tilde{\mu} + \half \Delta\tilde{t} \tilde{\vec{h}}\cdot\vec{\sigma})\tilde{\psi}} (\tilde{n}^2_a - \tilde{n}_a)} 
    }{
    \int DA\; e^{-\frac{1}{2} \sum_t A\transpose_t (-\Delta\tilde{t} \tilde{V})\inverse A_t - \frac{N_t}{2} \trlog{ -2\pi \Delta\tilde{t}\tilde{V}}}
    ~\tr{\prod_{t=1}^{N_t} e^{-\Delta \tilde{t}\tilde{K}}  e^{ \tilde{\psi}\adjoint (A_t + \Delta\tilde{t} \tilde{\mu} + \half \Delta\tilde{t} \tilde{\vec{h}}\cdot\vec{\sigma})\tilde{\psi}}} 
    }
\end{align}
which follows from a similar derivation as the bilinearized trotterization \eqref{trotterization bilinearized}.
By following the prescription demonstrated in our calculation of the particle number \eqref{fermionic particle number}, promoting the variation of chemical potential to a local field and differentiating, we find
\begin{align}
    \expectation{2\doubleOccupancy_a}
    &=
    \oneover{Z}\int DA e^{-\frac{1}{2} \sum_t A\transpose_t (-\Delta\tilde{t} \tilde{V})\inverse A_t - \frac{N_t}{2} \trlog{ -2\pi \Delta\tilde{t}\tilde{V}}}
    \left[\left(\oneover{\Delta t}\frac{\partial}{\partial \delta\mu_{t,a}}\right)^2 - \left(\oneover{\Delta t}\frac{\partial}{\partial \delta\mu_{t,a}}\right)\right]\tr{\prod_{t=1}^{N_t} e^{-\Delta \tilde{t}\tilde{K}}  e^{ \tilde{\psi}\adjoint (A_t + \Delta\tilde{t} \tilde{\mu} + \half \Delta\tilde{t} \tilde{\vec{h}}\cdot\vec{\sigma})\tilde{\psi}}}
    \nonumber\\
    &=
    \oneover{Z}\int DA e^{-\frac{1}{2} \sum_t A\transpose_t (-\Delta\tilde{t} \tilde{V})\inverse A_t - \frac{N_t}{2} \trlog{ -2\pi \Delta\tilde{t}\tilde{V}}}
    \left[\left(\oneover{\Delta t}\frac{\partial}{\partial \delta\mu_{t,a}}\right)^2 - \left(\oneover{\Delta t}\frac{\partial}{\partial \delta\mu_{t,a}}\right)\right] \det(\one + \UU)
    \nonumber\\
    &=
    \oneover{Z}\int DA e^{-\frac{1}{2} \sum_t A\transpose_t (-\Delta\tilde{t} \tilde{V})\inverse A_t - \frac{N_t}{2} \trlog{ -2\pi \Delta\tilde{t}\tilde{V}}}
    \left[\left(\oneover{\Delta t}\frac{\partial}{\partial \delta\mu_{t,a}}\right) - 1 \right] \left(\oneover{\Delta t}\frac{\partial}{\partial \delta\mu_{t,a}}\right) \det(\one + \UU)
    \nonumber\\
    &=
    \oneover{Z}\int DA e^{-\frac{1}{2} \sum_t A\transpose_t (-\Delta\tilde{t} \tilde{V})\inverse A_t - \frac{N_t}{2} \trlog{ -2\pi \Delta\tilde{t}\tilde{V}}}
    \left[\left(\oneover{\Delta t}\frac{\partial}{\partial \delta\mu_{t,a}}\right) - 1 \right] \det(\one + \UU) \tr{(\one + \UU)\inverse \UU \PP_a}
    \nonumber\\
    &=
    \oneover{Z}\int DA e^{-\frac{1}{2} \sum_t A\transpose_t (-\Delta\tilde{t} \tilde{V})\inverse A_t - \frac{N_t}{2} \trlog{ -2\pi \Delta\tilde{t}\tilde{V}}}
    \det( \one+\UU ) \left(\tr{(\one + \UU)\inverse \UU \PP_a}^2 - \tr{(\one+\UU)\inverse \UU \PP_a (\one+\UU)\inverse \UU \PP_a}\right)
    \nonumber\\
    \expectation{\doubleOccupancy_a}
    &=
    \half\expectation{\tr{(\one + \UU)\inverse \UU \PP_a}^2 - \tr{(\one+\UU)\inverse \UU \PP_a (\one+\UU)\inverse \UU \PP_a}}
    \label{eq:double occupancy}
\end{align}
where the right-hand side expectation value is a grand-canonical auxiliary-field expression.
We can rewrite these projected traces if we leverage the simple properties of the projector,
\begin{align}
    \expectation{\doubleOccupancy_a}
    &=
    \half\expectation{\sum_{st}\; [(\one + \UU)\inverse \UU]_{aa}^{ss} [(\one + \UU)\inverse \UU]_{aa}^{tt} - [(\one + \UU)\inverse \UU]_{aa}^{st} [(\one + \UU)\inverse \UU]_{aa}^{ts} }
    \nonumber\\
    &=
    \half\expectation{\sum_{st}\; \G_{aa}^{ss} \G_{aa}^{tt} - \G_{aa}^{ts} \G_{aa}^{st} }
\end{align}
where $s,t$ are spin indices; this form is easily amenable to programming via \href{https://pytorch.org/docs/stable/generated/torch.einsum.html}{\texttt{einsum}}.

We can also proceed via Wick contractions.
Start with the operator definition and get it into normal order,
\begin{align}
	2\doubleOccupancy = n^2 - n
	&=
	\tilde{\psi}\adjoint_{xs}\tilde{\psi}_{xs}
	\tilde{\psi}\adjoint_{xt}\tilde{\psi}_{xt}
	-
	\tilde{\psi}\adjoint_{xs}\tilde{\psi}_{xs}
	\nonumber\\
	&=
	\tilde{\psi}\adjoint_{xs}
	\left( \delta_{xx}\delta_{st} - \tilde{\psi}\adjoint_{xt}\tilde{\psi}_{xs} \right)
	\tilde{\psi}_{xt}
	-
	\tilde{\psi}\adjoint_{xs}\tilde{\psi}_{xs}
	\nonumber\\
	&=
	-\tilde{\psi}\adjoint_{xs}\tilde{\psi}\adjoint_{xt}\tilde{\psi}_{xs}\tilde{\psi}_{xt}
	\nonumber\\
	&=
	\tilde{\psi}\adjoint_{xt}\tilde{\psi}\adjoint_{xs}\tilde{\psi}_{xs}\tilde{\psi}_{xt}
\end{align}
Now we perform Wick contractions,
\begin{align}
	2\doubleOccupancy_a &= 
	\contraction[1em]{}{\tilde{\psi}\adjoint_{at}}{\tilde{\psi}\adjoint_{as}\tilde{\psi}_{as}}{\tilde{\psi}_{at}}
	\contraction[0.75em]{\tilde{\psi}\adjoint_{at}}{\tilde{\psi}\adjoint_{as}}{}{\tilde{\psi}_{as}}
	\tilde{\psi}\adjoint_{at}\tilde{\psi}\adjoint_{as}\tilde{\psi}_{as}\tilde{\psi}_{at}
	+
	\contraction[1em]{}{\tilde{\psi}\adjoint_{at}}{\tilde{\psi}\adjoint_{as}}{\tilde{\psi}_{as}}
	\contraction[0.75em]{\tilde{\psi}\adjoint_{at}}{\tilde{\psi}\adjoint_{as}}{\tilde{\psi}_{as}}{\tilde{\psi}_{at}}
	\tilde{\psi}\adjoint_{at}\tilde{\psi}\adjoint_{as}\tilde{\psi}_{as}\tilde{\psi}_{at}
	\nonumber\\
	&=
	\G^{tt}_{aa}
	\G^{ss}_{aa}
	-
	\G^{ts}_{aa}
	\G^{st}_{aa}
\end{align}
which matches the other derivation \eqref{double occupancy}.
The minus sign on the second term comes from the crossing in the second contraction.

\subsubsection{The $\beta\goesto0$ Limit}

One way to determine the double occupancy in the infinite temperature limit is to just enumerate states.
On a given site the possible states are $\ket{\null}$, $\ket{\up}$, $\ket{\dn}$, and $\ket{\up\dn}$ and only the last has nonnzero double occupancy, so the expected value is $\oneover{4}$.
Using the auxiliary field expression for double occupancy \eqref{double occupancy} and taking the $\beta\goesto0$ limit as in \secref{infinite temperature} one finds
\begin{align}
    \expectation{\doubleOccupancy_a}
    \goesto&
    \half\expectation{\tr{(\one + \one)\inverse \one \PP_a}^2 - \tr{(\one+\one)\inverse \one \PP_a (\one+\one)\inverse \one \PP_a}}
    \nonumber\\
    =&
    \half\expectation{\tr{\oneover{2} \PP_a}^2 - \tr{\oneover{2} \PP_a \oneover{2} \PP_a}}
    \nonumber\\
    =&
    \oneover{8}\expectation{\tr{\PP_a}^2 - \tr{\PP_a}}
    =
    \oneover{8}\expectation{ 2^2 - 2 } = \frac{2}{8} = \frac{1}{4}.
\end{align}
where we recall that the projector $\PP_a$ projects both spins to site $a$.

In a canonical sector we need to evaluate
\begin{align}
    \expectation{ \doubleOccupancy_a \PP_{N,S_h}} \expectation{\PP_{N,S_h}}
    =
    \half \Bigg\langle &\oneover{(2\Volume+1)^2}\oneover{\det(\one+\UU)} \sum_{ns} e^{-\frac{2\pi i}{2\Volume+1}(nN+2sS_h)} \det(\one+\UU_{ns})
    \nonumber\\
    &\left\{\tr{(\one + \UU_{ns})\inverse \UU_{ns} \PP_a}^2 - \tr{(\one+\UU_{ns})\inverse \UU_{ns} \PP_a (\one+\UU_{ns})\inverse \UU_{ns} \PP_a}\right\} \Bigg\rangle
    \nonumber\\
    =\Bigg\langle & \oneover{(2\Volume+1)^2} \oneover{2^{2\Volume+1}} \sum_{ns} e^{-\frac{2\pi i}{2\Volume+1}(nN+2sS_h)} \left(1+e^{\frac{2\pi i}{2\Volume+1}(n+s)}\right)^\Volume \left(1+e^{\frac{2\pi i}{2\Volume+1}(n-s)}\right)^\Volume
    \nonumber\\
    &\left\{
        \left(
            \frac{e^{\frac{2\pi i}{2\Volume+1}(n+s)}}{1+e^{\frac{2\pi i}{2\Volume+1}(n+s)}}
          + \frac{e^{\frac{2\pi i}{2\Volume+1}(n-s)}}{1+e^{\frac{2\pi i}{2\Volume+1}(n-s)}}
        \right)^2
      - \left(
            \left[\frac{e^{\frac{2\pi i}{2\Volume+1}(n+s)}}{1+e^{\frac{2\pi i}{2\Volume+1}(n+s)}}\right]^2
          + \left[\frac{e^{\frac{2\pi i}{2\Volume+1}(n-s)}}{1+e^{\frac{2\pi i}{2\Volume+1}(n-s)}}\right]^2
        \right)
      \right\}
      \Bigg\rangle
    \nonumber\\
    =\Bigg\langle & \oneover{(2\Volume+1)^2} \oneover{2^{2\Volume+1}} \sum_{ns} e^{-\frac{2\pi i}{2\Volume+1}(nN+2sS_h)} \left(1+e^{\frac{2\pi i}{2\Volume+1}(n+s)}\right)^\Volume \left(1+e^{\frac{2\pi i}{2\Volume+1}(n-s)}\right)^\Volume
    \nonumber\\
    &\left\{
        \frac{2 e^{\frac{2\pi i}{2\Volume+1} \cdot 2n}}{
            \left(1+e^{\frac{2\pi i}{2\Volume+1}(n+s)}\right)
            \left(1+e^{\frac{2\pi i}{2\Volume+1}(n-s)}\right)
        }
      \right\}
      \Bigg\rangle
    \nonumber\\
    =\Bigg\langle & \oneover{(2\Volume+1)^2} \oneover{2^{2\Volume}} \sum_{ns} e^{-\frac{2\pi i}{2\Volume+1}(nN+2sS_h)} \left(1+e^{\frac{2\pi i}{2\Volume+1}(n+s)}\right)^{\Volume-1} \left(1+e^{\frac{2\pi i}{2\Volume+1}(n-s)}\right)^{\Volume-1} e^{\frac{2\pi i}{2\Volume+1} \cdot 2n} \Bigg\rangle
    \nonumber\\
    =\Bigg\langle & \oneover{(2\Volume+1)^2} \oneover{2^{2\Volume}} \sum_{nsab} \begin{pmatrix} \Volume-1 \\ a \end{pmatrix} \begin{pmatrix} \Volume-1 \\ b \end{pmatrix} e^{\frac{2\pi i}{2\Volume+1} n (2+a+b-N)} e^{\frac{2\pi i}{2\Volume+1} s (a-b-2S_h)} \Bigg\rangle
    \nonumber\\
    =\Bigg\langle & \oneover{2^{2\Volume}} \sum_{ab} \begin{pmatrix} \Volume-1 \\ a \end{pmatrix} \begin{pmatrix} \Volume-1 \\ b \end{pmatrix} \delta_{N,a+b+2} \delta_{2S_h,a-b} \Bigg\rangle
    \nonumber\\
    &= \oneover{2^{2\Volume}} \begin{pmatrix} \Volume-1 \\ \frac{N}{2}-1+S_h \end{pmatrix} \begin{pmatrix} \Volume-1 \\ \frac{N}{2}-1-S_h \end{pmatrix}
\end{align}
so that dividing by the canonical infinite-temperature partition function \eqref{canonical infinite-temperature partition function} one finds
\begin{align}
    \expectation{\doubleOccupancy_a \PP_{N,S_h}}
    =
    \frac{
        \begin{pmatrix} \Volume-1 \\ \frac{N}{2}-1+S_h \end{pmatrix} \begin{pmatrix} \Volume-1 \\ \frac{N}{2}-1-S_h \end{pmatrix}
    }{  \begin{pmatrix} \Volume \\ \frac{N}{2} + S_h \end{pmatrix} \begin{pmatrix} \Volume \\ \frac{N}{2}-S_h \end{pmatrix}}
    =
    \oneover{\Volume^2}\left(\frac{N}{2}+S_h\right)\left(\frac{N}{2}-S_h\right)
\end{align}
so that the total double occupancy 
\begin{align}
    \expectation{\DoubleOccupancy \PP_{N,S_h}}
    =
    \expectation{\sum_a \doubleOccupancy_a\PP_{N,S_h}}
    =
    \oneover{\Volume}\left(\frac{N}{2}+S_h\right)\left(\frac{N}{2}-S_h\right)
    \label{eq:infinite temperature Double Occupancy}
\end{align}
which matches some simple examples we can easily do by counting and extreme limits.
For example, the double occupancy vanishes when $\abs{S_h}$ is maximal, which makes sense via the Pauli exclusion principle.
When $N=2$ and $S_h=0$, the spin-\up\ particle can be anywhere and the odds that the spin-\dn\ particle is on the same site is $\Volume\inverse$, which agrees with the formula.
A more nontrivial example is to calculate the double occupancy with $N=4$ and $S_h=0$ via a counting argument:
\begin{align}
    \expectation{\DoubleOccupancy \PP_{N,S_h}}
    = \oneover{\begin{pmatrix} \Volume \\ 2 \end{pmatrix}^2}\left[
        0 \times \begin{pmatrix} \Volume \\ 2 \end{pmatrix} \begin{pmatrix} \Volume-2 \\ 2 \end{pmatrix}
      + 1 \times \begin{pmatrix} \Volume \\ 2 \end{pmatrix} \begin{pmatrix} 2 \\ 1 \end{pmatrix} \begin{pmatrix} \Volume-2 \\ 1 \end{pmatrix}
      + 2 \times \begin{pmatrix} \Volume \\ 2 \end{pmatrix} \begin{pmatrix} 2 \\ 2 \end{pmatrix} \begin{pmatrix} \Volume-2 \\ 0 \end{pmatrix}
      \right]
    = \frac{4}{\Volume}
\end{align}
where each term counts the number of ways to put two spin-\up{}s arbitrarily, two spin-\dn{}s so that they have 0, 1, or 2 sites in common with the spin-\up{}s, weighted by the number of sites in common, and the result matches the infinite temperature double occupancy \eqref{infinite temperature Double Occupancy}.
