\subsection{Vortex-Sensitive Observables}\label{sec:vortex}

The BKT transition is a transition between no vortices (at low temperature) and lots of vortex/antivortex pairs (at high temperature), which is driven by the entropic considerations of where all the vortices can be~\cite{1971JETP...32..493B,1972JETP...34..610B,Kosterlitz:1973}.
\Ref{PhysRevLett.129.076403} provides an analysis of the BKT transition by studying the condensate fraction [their (2)] and the Cooper pair momentum distribution [their (3)].
This transition is topological, in the sense that there isn't a good local order parameter to capture the change we're talking about.
Consider, for example, an observable that measures the local vorticity, like
\begin{align}
	\tilde{\omega}^3_x = \text{curl}(\tilde{j})_x = \tilde{\grad} \times \tilde{j}_x = \epsilon^{3mn} \tilde{\partial}^m \tilde{j}^n_x
\end{align}
which we can simplify using a simple form of the current \eqref{current hermitian}
\begin{align}
	\tilde{\omega}^3_x
	&=
	\epsilon^{3mn} \tilde{\partial}^m \oneover{2\Volume}
	\sum_{qk} e^{+2\pi i (q-k) x / N_x} \left(\frac{2\pi [ k+q ]^n}{N_x}\right)
	\tilde{\psi}\adjoint_{k} \tilde{\psi}_{q}
	\nonumber\\
	&=
	\oneover{2\Volume}
	\sum_{qk} e^{+2\pi i (q-k) x / N_x} i \left(\frac{2\pi}{N_x}\right)^2 \epsilon^{3mn} [q-k]^m[ k+q ]^n
	\tilde{\psi}\adjoint_{k} \tilde{\psi}_{q}
	\nonumber\\
	&=
	\frac{i}{\Volume}
	\sum_{qk} e^{+2\pi i (q-k) x / N_x} \left(\frac{2\pi}{N_x}\right)^2 \epsilon^{3mn} q^m k^n
	\tilde{\psi}\adjoint_{k} \tilde{\psi}_{q}
	\nonumber\\
	&=
	\frac{i}{\Volume}
	\sum_{qk} e^{+2\pi i (q-k) x / N_x} \left(\frac{2\pi}{N_x}\right)^2 (q \cross k)\,
	\tilde{\psi}\adjoint_{k} \tilde{\psi}_{q}.
	\label{eq:vorticity}
\end{align}
The physical $\omega = \grad \cross j$ [both physical] and therefore trading the physical $j$ for $\tilde{j}$ \eqref{j physical} and using $\tilde{\grad} = \Delta x \grad$ we immediately have
\begin{align}
	\omega = \frac{(\tilde{\grad} \cross \tilde{j} = \tilde{\omega})}{ML^2 \Delta x^2}.
	\label{eq:omega physical}
\end{align}

The vorticity is Hermitian,
\begin{align}
	\tilde{\omega}^3\adjoint_x
	&=
	-\frac{i}{\Volume}
	\sum_{qk} e^{-2\pi i (q-k) x / N_x} \left(\frac{2\pi}{N_x}\right)^2 (q \cross k)\,
	\tilde{\psi}\adjoint_{q} \tilde{\psi}_{k}
	\nonumber\\
	&=
	\frac{i}{\Volume}
	\sum_{qk} e^{+2\pi i (k-q) x / N_x} \left(\frac{2\pi}{N_x}\right)^2 (-q \cross k = k \cross q)\,
	\tilde{\psi}\adjoint_{q} \tilde{\psi}_{k}
	=
	\tilde{\omega}^3_x
\end{align}
and is a local observable.
By periodic boundary conditions the net vorticity vanishes
\begin{align}
	\sum_x \tilde{\omega}^3_x 
	&=
	\frac{i}{\Volume}
	\sum_{qk} \left(\frac{2\pi}{N_x}\right)^2
	(q \cross k)\,
	\tilde{\psi}\adjoint_{k} \tilde{\psi}_{q}
	\left(\sum_x e^{+2\pi i (q-k) x / N_x} = \Volume \delta_{kq}\right)
	\propto
	k \cross k
	=
	0
	\label{eq:net vorticity}
\end{align}
if the current $\tilde{j}$ corresponds obeys a lattice continuity equation, since particles cannot leave across a boundary without coming back `on the other side'.
(The vorticity is entirely in the 3 direction, since the fermions are confined to 2D; in what follows we suppress the 3 superscript.)
This vanishes exactly because of the periodic boundary conditions; the boundary-condition-dependence is a hint it's accessing something topological.

Instead of the total vorticity \eqref{net vorticity}, which must vanish by periodic boundary conditions, we can consider the (local) square,
\begin{align}
	\tilde{\omega}^2_x = \tilde{\omega}\adjoint_x \tilde{\omega}_x
	=
	\oneover{\Volume^2} \sum_{q'k'qk} e^{+2\pi i[ (q-k) - (q'-k') ] x / N_x}
	\left(\frac{2\pi}{N_x}\right)^4 (q' \cross k') \cdot (q \cross k)\,
	\tilde{\psi}\adjoint_{s'q'} \tilde{\psi}_{s'k'} \tilde{\psi}\adjoint_{sk} \tilde{\psi}_{sq}.
\end{align}
which is positive (semi)definite.
Unfortunately, it's EXTREMELY positive.
In fact, according to an operator product expansion analysis, it is infinite and diverges in the continuum limit.

We could instead consider the vorticity-vorticity two-point correlation
\begin{align}
	\tilde{\Omega}_{r} &= \oneover{\Volume} \sum_x \tilde{\Omega}_{x,x-r}
	&
	\tilde{\Omega}_{xy} &= \tilde{\omega}\adjoint_x \tilde{\omega}_y,
\end{align}
which at zero displacement is the average square vorticity $\tilde{\Omega}_0 = \Volume\inverse \sum_x \tilde{\omega}^2_x$.
In the continuum we would analogously define
\begin{align}
	\Omega(r) = \oneover{L^2} \int d^2x \left[\Omega(x,x-r) = \omega(x) \omega(x-r) \right]
\end{align}
using the physical $\omega$ \eqref{omega physical}.
The lattice quantity $\tilde{\Omega}$ converges to the fixed-volume continuum limit according to
\begin{align}
	\lim_{\Delta x \goesto 0} \left(\oneover{ML^2 \Delta x^2}\right)^2 \tilde{\Omega}_r \goesto \Omega(r).
\end{align}
Because it matches the local square at $r=0$ its value there diverges according to the same OPE argument.

If there were one vortex/antivortex pair that were well-separated the correlations would reach far as a function of the separation $\delta$.
But when there are very many vortices and antivortices $\Omega$ should decay quickly.
\begin{align}
	\tilde{\Omega}_{r}
	&=
	\oneover{\Volume} \sum_{x}
	\oneover{\Volume^2} \sum_{q'k'qk}
		e^{-2\pi i(q'-k')x/N_x} \left(\frac{2\pi}{N_x}\right)^2 (q' \cross k')\, \tilde{\psi}\adjoint_{q'} \tilde{\psi}_{k'}
		e^{+2\pi i(q-k)(x-r)/N_x} \left(\frac{2\pi}{N_x}\right)^2 (q \cross k)\, \tilde{\psi}\adjoint_{k} \tilde{\psi}_q
	\nonumber\\
	&=
	\oneover{\Volume^2} \sum_{q'k'qk}
		e^{-2\pi i(q-k)r/N_x} \left(\frac{2\pi}{N_x}\right)^4 (q' \cross k')\cdot(q \cross k)\,
		\tilde{\psi}\adjoint_{q'} \tilde{\psi}_{k'}
		\tilde{\psi}\adjoint_{k} \tilde{\psi}_q
		\left(\oneover{\Volume} \sum_{x}
			e^{+2\pi i(q-k-q'+k') x / N_x}
			= \delta_{q-k,q'-k'}
			\right)
\end{align}
where the dot product between the cross products arises naturally from considering $\omega\adjoint \cdot \omega$ (and we focus on the $z$ direction when the momenta are all confined to the plane).
We now shift $q' = p+k'$, so that $p=q'-k'$ and $q'\cross k' = (p'+k')\cross k' = p' \cross k'$.
\begin{align}
	\tilde{\Omega}_{r}
	&=
	\oneover{\Volume^2} \sum_{pk'qk}
		e^{-2\pi ipr/N_x} \left(\frac{2\pi}{N_x}\right)^4 (p \cross k')\cdot(q \cross k)\,
		\tilde{\psi}\adjoint_{p+k'} \tilde{\psi}_{k'}
		\tilde{\psi}\adjoint_{k} \tilde{\psi}_q
		\delta_{q-k,p}
	\nonumber\\
	&=
	\oneover{\Volume^2} \sum_{pk'k}
		e^{-2\pi ipr/N_x} \left(\frac{2\pi}{N_x}\right)^4 (p \cross k')\cdot(p \cross k)\,
		\tilde{\psi}\adjoint_{p+k'} \tilde{\psi}_{k'}
		\tilde{\psi}\adjoint_{k} \tilde{\psi}_{p+k},
	\label{eq:vorticity-vorticity}
\end{align}
executing the sum over $q$ and using $(p+k)\cross k = p\cross k$.
This may be reexpressed using a vector identity $(p \cross k')\cdot(p \cross k) = p^2 k' \cdot k - (p \cdot k')(p \cdot k)$ but need not be.
The vortex susceptibility is $\tilde{\Omega}_0$, which should (presumably?) show the usual susceptibility behavior at the BKT transition.

Because $\Omega(r=0)$ diverges and $\Omega(k=0)=0$ we define
\begin{align}
	B_n(k) = \int d^2r\; e^{-i k r} \abs{r}^n \Omega(r) 
\end{align}
Periodic boundary conditions guarantees that $B_0(0) = 0$.
The dimensions of $B_n$ are $[M^{-2} L^{-(6-n)}]$, and we can eliminate $L$ dependence with appropriate powers of $k_F$.
We write the $k=0$ values as lower case.
So 
\begin{align}
	\frac{M^2 b_2}{k_F^4} &= \frac{\sum_r r^2 \tilde{\Omega}_r}{(2\pi N)^2}
	&
	\frac{M^2 b_4}{k_F^2} &= \oneover{\Volume} \frac{\sum_r r^4 \tilde{\Omega}_r}{(2\pi N)}
	&
	M^2 b_6 &= \oneover{\Volume^2}\sum_r r^6 \tilde{\Omega}_r,
	\label{eq:voricity moments}
\end{align}
where $r$ are integer pairs, all have good continuum limits.
The third nonzero even moment $b_6$ requires no powers of $k_F^4$ and therefore feels mysterious.
We give an example calculation of \incode{docs/sanity-checks/free-theorty-vorticity-continuum-limit.ipynb}{the free-theory continuum extrapolation}, which shows a staggered behavior based on $N_x\mod{4}$.
