\section{Observables}\label{sec:observables}

\subsection{Total Particle Number}\label{sec:number}

In \Secref{correlation functions} we give an example for calculating one-body observables by functionally differentiating the partition function.
In particular, by calculating a trace of combinations of $\UU$ one may calculate the local particle number $\expectation{\tilde{n}_a}$ \eqref{fermionic particle number}.
By summing the result on every lattice site we find
\begin{align}
	\expectation{\tilde{N}}
	=
	\expectation{\sum_a \tilde{n}_a}
	=
	\sum_a \expectation{\tilde{n}_a}
	&=
	\sum_a \expectation{\tr{(\one+\UU)\inverse \UU \PP_a}}
	\nonumber\\
	&=
	\expectation{\tr{(\one+\UU)\inverse \UU \sum_a \PP_a}}
	=
	\expectation{\tr{(\one+\UU)\inverse \UU \one}}
	=
	\expectation{\tr{(\one+\UU)\inverse \UU}}
	\label{eq:fermionic total number}
\end{align}
Since this observable requires the fermion matrix $\UU$, let's call this the fermionic method for computing $\tilde{N}$.

We will now show that there is a much cheaper estimator for $\tilde{N}$.\footnote{This point is strongly informed by discussions with Andreas Wipf.}
The philosophy is similar to the derivation of a Ward identity.
The starting point is the bilinear Trotterization \eqref{trotterization bilinearized}.
Consider shifting each auxiliary field timeslice in the numerator by any finite amount $\delta A_t$ (which might vary in space).
Since $A$ is integrated over the real line such a shift cannot change the partition function.
Therefore if we expand the result order-by-order in $\delta A$ we will get a series of terms that all must vanish.

Concretely,
\begin{align}
	\int DA&\; e^{-\frac{1}{2} \sum_t A\transpose_t (-\Delta\tilde{t} \tilde{V})\inverse A_t - \frac{N_t}{2} \trlog{-2\pi \Delta\tilde{t}\tilde{V}}}
		\tr{\prod_{t=1}^{N_t} e^{-\Delta \tilde{t}\tilde{K}}  e^{ \tilde{\psi}\adjoint (A_t + \Delta\tilde{t} \tilde{\mu} + \Delta\tilde{t} \tilde{\vec{h}}\cdot\vec{\sigma})\tilde{\psi}} } 
	\nonumber\\
	=&
	\int DA\; e^{-\frac{1}{2} \sum_t (A+\delta A)\transpose_t (-\Delta\tilde{t} \tilde{V})\inverse (A+\delta A)_t - \frac{N_t}{2} \trlog{-2\pi \Delta\tilde{t}\tilde{V}}}
	~	\tr{\prod_{t=1}^{N_t} e^{-\Delta \tilde{t}\tilde{K}}  e^{ \tilde{\psi}\adjoint ((A+\delta A)_t + \Delta\tilde{t} \tilde{\mu} + \Delta\tilde{t} \tilde{\vec{h}}\cdot\vec{\sigma})\tilde{\psi}} } 
	\nonumber\\
	=&
	\int DA\; e^{-\frac{1}{2} \sum_t A\transpose_t (-\Delta\tilde{t} \tilde{V})\inverse A_t - \frac{N_t}{2}\trlog{-2\pi \Delta\tilde{t}\tilde{V}}}
		\left\{1 - \half \sum_t \left(A\transpose_t (-\Delta\tilde{t}\tilde{V})\inverse \delta A_t + \delta A\transpose_t (-\Delta\tilde{t}\tilde{V})\inverse A_t\right) +\cdots\right\} 
	\nonumber\\
	&\times\tr{\prod_{t=1}^{N_t} e^{-\Delta \tilde{t}\tilde{K}}  e^{ \tilde{\psi}\adjoint (A + \Delta\tilde{t} \tilde{\mu} + \Delta\tilde{t} \tilde{\vec{h}}\cdot\vec{\sigma})\tilde{\psi}} \left\{1 + \delta A_t \tilde{\psi}\adjoint_t \tilde{\psi}_t + \cdots\right\} } 
\end{align}
Collecting the right-hand-side order-by-order in $\delta A$ gives the original expression at the $0^\text{th}$ order, so all further orders must be individually 0.
In particular, the first order in $\delta A$ gives
\begin{align}
	0 =
	\delta A_t \Bigg(&
		\int DA\; e^{-\frac{1}{2} \sum_t A\transpose_t (-\Delta\tilde{t} \tilde{V})\inverse A_t - \frac{N_t}{2}\trlog{-2\pi \Delta\tilde{t}\tilde{V}}}
		\tr{\prod_{t=1}^{N_t} e^{-\Delta \tilde{t}\tilde{K}}  e^{ \tilde{\psi}\adjoint (A + \Delta\tilde{t} \tilde{\mu} + \Delta\tilde{t} \tilde{\vec{h}}\cdot\vec{\sigma})\tilde{\psi}} \tilde{\psi}\adjoint_t \tilde{\psi}_t }
	\nonumber\\
	&
	-	\int DA\; e^{-\frac{1}{2} \sum_t A\transpose_t (-\Delta\tilde{t} \tilde{V})\inverse A_t - \frac{N_t}{2}\trlog{-2\pi \Delta\tilde{t}\tilde{V}}}
		\tr{\prod_{t=1}^{N_t} e^{-\Delta \tilde{t}\tilde{K}}  e^{ \tilde{\psi}\adjoint (A + \Delta\tilde{t} \tilde{\mu} + \Delta\tilde{t} \tilde{\vec{h}}\cdot\vec{\sigma})\tilde{\psi}} } \left((-\Delta\tilde{t}\tilde{V})\inverse A_t\right)
	\Bigg) 
	\nonumber\\
	0 = \delta A_t \Bigg(& \expectation{\tilde{\psi}\adjoint_t\tilde{\psi}_t = \tilde{n}_t} - \expectation{ (-\Delta\tilde{t}\tilde{V})\inverse A_t} \Bigg)
\end{align}
where we used the symmetry of $\tilde{V}$ to simplify the gauge piece and conclude that for each site $a$ we can estimate
\begin{align}
	\expectation{n_{ta}} &= \expectation{(-\Delta\tilde{t}\tilde{V})_{ab}\inverse A_{tb}}
\end{align}
To compute the total particle number we can simply sum over $a$; for improved statistics we can average over time $t$,
\begin{align}
	\expectation{\tilde{N}} &= \expectation{\oneover{N_t}\sum_{ta}(-\Delta\tilde{t}\tilde{V})_{ab}\inverse A_{tb}}
	\label{eq:bosonic particle number}
\end{align}
which gives a purely bosonic estimator that can be computed from the auxiliary field very quickly.

This worked because we performed the Hubbard-Stratanovich transformation in the density channel.
We can go to higher order in $\delta A$ with greater analytic effort, though at higher orders it is not possible to eliminate the fermionic operators completely: it is only possible to reduce the number of bilinears by 1.

We note that the computational speed of the bosonic estimator is a tradeoff.
In a few tests at large particle number the bosonic and fermionic estimators yielded similar distributions.
However, the fermionic estimator \eqref{fermionic particle number} seems to be positive-definite in tests at small particle number, while the bosonic estimator is not.
Therefore, especially at small particle number, the bosonic method may have much greater variance.

\subsection{The Contact}\label{sec:contact}

The contact is given by~\cite{PhysRevA.86.013626}
\begin{align}
	\hat{C} &= \frac{2 \pi M}{\hbar^2} \frac{dH}{d\log a}
	&
	\hat{C}\Delta x^2 &= 2\pi \frac{d\tilde{H}}{d\log a} 
	\label{eq:contact}
\end{align}
\Refs{Froehlich2011,PhysRevLett.109.130403} report a measurement of the contact.
Specifically they measure $C/k_F^2$ as a function of $1/\log k_F a_{2D}$ in the Fermi liquid regime.
In \Ref{PhysRevLett.109.130403} Fig. 4 they show a comparison with a zero-temperature fixed-node diffusion Monte Carlo calculation~\cite{PhysRevLett.106.110403}; the theory and experiment agree at $1/\log k_F a \gtrsim 0.5$ but the experiment and the theory diverge deeper into the weak-coupling regime $1/\ln k_F a \sim 0.3$.
The disagreement is only one data point, so who knows; the experimentalists did a ladder approximation and reproduced that one point.
I can't make heads or tails of the method; \Ref{doi:10.1063/1.443766} claims the solution is exact in the region bounded by the nodes.
