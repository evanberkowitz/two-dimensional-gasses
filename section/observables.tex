\section{Observables}\label{sec:observables}

\subsection{Total Particle Number}\label{sec:number}

In \Secref{correlation functions} we give an example for calculating one-body observables by functionally differentiating the partition function.
In particular, by calculating a trace of combinations of $\UU$ one may calculate the local particle number $\expectation{\tilde{n}_a}$ \eqref{fermionic particle number}.
By summing the result on every lattice site we find
\begin{align}
	\expectation{\tilde{N}}
	=
	\expectation{\sum_a \tilde{n}_a}
	=
	\sum_a \expectation{\tilde{n}_a}
	&=
	\sum_a \expectation{\tr{(\one+\UU)\inverse \UU \PP_a}}
	\nonumber\\
	&=
	\expectation{\tr{(\one+\UU)\inverse \UU \sum_a \PP_a}}
	=
	\expectation{\tr{(\one+\UU)\inverse \UU \one}}
	=
	\expectation{\tr{(\one+\UU)\inverse \UU}}
	\label{eq:fermionic total number}
\end{align}
Since this observable requires the fermion matrix $\UU$, let's call this the fermionic method for computing $\tilde{N}$.

We will now show that there is a much cheaper estimator for $\tilde{N}$.\footnote{This point is strongly informed by discussions with Andreas Wipf.}
The philosophy is similar to the derivation of a Ward identity.
The starting point is the bilinear Trotterization \eqref{trotterization bilinearized}.
Consider shifting each auxiliary field timeslice in the numerator by any finite amount $\delta A_t$ (which might vary in space).
Since $A$ is integrated over the real line such a shift cannot change the partition function.
Therefore if we expand the result order-by-order in $\delta A$ we will get a series of terms that all must vanish.

Concretely,
\begin{align}
	\int DA&\; e^{-\frac{1}{2} \sum_t A\transpose_t (-\Delta\tilde{t} \tilde{V})\inverse A_t - \frac{N_t}{2} \trlog{-2\pi \Delta\tilde{t}\tilde{V}}}
		\tr{\prod_{t=1}^{N_t} e^{-\Delta \tilde{t}\tilde{K}}  e^{ \tilde{\psi}\adjoint (A_t + \Delta\tilde{t} \tilde{\mu} + \Delta\tilde{t} \tilde{\vec{h}}\cdot\vec{\sigma})\tilde{\psi}} } 
	\nonumber\\
	=&
	\int DA\; e^{-\frac{1}{2} \sum_t (A+\delta A)\transpose_t (-\Delta\tilde{t} \tilde{V})\inverse (A+\delta A)_t - \frac{N_t}{2} \trlog{-2\pi \Delta\tilde{t}\tilde{V}}}
	~	\tr{\prod_{t=1}^{N_t} e^{-\Delta \tilde{t}\tilde{K}}  e^{ \tilde{\psi}\adjoint ((A+\delta A)_t + \Delta\tilde{t} \tilde{\mu} + \Delta\tilde{t} \tilde{\vec{h}}\cdot\vec{\sigma})\tilde{\psi}} } 
	\nonumber\\
	=&
	\int DA\; e^{-\frac{1}{2} \sum_t A\transpose_t (-\Delta\tilde{t} \tilde{V})\inverse A_t - \frac{N_t}{2}\trlog{-2\pi \Delta\tilde{t}\tilde{V}}}
		\left\{1 - \half \sum_t \left(A\transpose_t (-\Delta\tilde{t}\tilde{V})\inverse \delta A_t + \delta A\transpose_t (-\Delta\tilde{t}\tilde{V})\inverse A_t\right) +\cdots\right\} 
	\nonumber\\
	&\times\tr{\prod_{t=1}^{N_t} e^{-\Delta \tilde{t}\tilde{K}}  e^{ \tilde{\psi}\adjoint (A + \Delta\tilde{t} \tilde{\mu} + \Delta\tilde{t} \tilde{\vec{h}}\cdot\vec{\sigma})\tilde{\psi}} \left\{1 + \delta A_t \tilde{\psi}\adjoint_t \tilde{\psi}_t + \cdots\right\} } 
\end{align}
Collecting the right-hand-side order-by-order in $\delta A$ gives the original expression at the $0^\text{th}$ order, so all further orders must be individually 0.
In particular, the first order in $\delta A$ gives
\begin{align}
	0 =
	\delta A_t \Bigg(&
		\int DA\; e^{-\frac{1}{2} \sum_t A\transpose_t (-\Delta\tilde{t} \tilde{V})\inverse A_t - \frac{N_t}{2}\trlog{-2\pi \Delta\tilde{t}\tilde{V}}}
		\tr{\prod_{t=1}^{N_t} e^{-\Delta \tilde{t}\tilde{K}}  e^{ \tilde{\psi}\adjoint (A + \Delta\tilde{t} \tilde{\mu} + \Delta\tilde{t} \tilde{\vec{h}}\cdot\vec{\sigma})\tilde{\psi}} \tilde{\psi}\adjoint_t \tilde{\psi}_t }
	\nonumber\\
	&
	-	\int DA\; e^{-\frac{1}{2} \sum_t A\transpose_t (-\Delta\tilde{t} \tilde{V})\inverse A_t - \frac{N_t}{2}\trlog{-2\pi \Delta\tilde{t}\tilde{V}}}
		\tr{\prod_{t=1}^{N_t} e^{-\Delta \tilde{t}\tilde{K}}  e^{ \tilde{\psi}\adjoint (A + \Delta\tilde{t} \tilde{\mu} + \Delta\tilde{t} \tilde{\vec{h}}\cdot\vec{\sigma})\tilde{\psi}} } \left((-\Delta\tilde{t}\tilde{V})\inverse A_t\right)
	\Bigg) 
	\nonumber\\
	0 = \delta A_t \Bigg(& \expectation{\tilde{\psi}\adjoint_t\tilde{\psi}_t = \tilde{n}_t} - \expectation{ (-\Delta\tilde{t}\tilde{V})\inverse A_t} \Bigg)
\end{align}
where we used the symmetry of $\tilde{V}$ to simplify the gauge piece and conclude that for each site $a$ we can estimate
\begin{align}
	\expectation{n_{ta}} &= \expectation{(-\Delta\tilde{t}\tilde{V})_{ab}\inverse A_{tb}}
\end{align}
To compute the total particle number we can simply sum over $a$; for improved statistics we can average over time $t$,
\begin{align}
	\expectation{\tilde{N}}
	= \expectation{\oneover{N_t}\sum_{ta}(-\Delta\tilde{t}\tilde{V})_{ab}\inverse A_{tb}}
	= - \oneover{N_t \Delta t} \sum_a \tilde{V}_{ab}\inverse \expectation{ \sum_t A_{tb}}
	= - \oneover{\beta} \sum_a \tilde{V}_{ab}\inverse \expectation{ \sum_t A_{tb}}
	\label{eq:bosonic particle number}
\end{align}
which gives a purely bosonic estimator that can be computed from the auxiliary field very quickly.

This worked because we performed the Hubbard-Stratanovich transformation in the density channel.
We can go to higher order in $\delta A$ with greater analytic effort, though at higher orders it is not possible to eliminate the fermionic operators completely: it is only possible to reduce the number of bilinears by 1.

We note that the computational speed of the bosonic estimator is a tradeoff.
In a few tests at large particle number the bosonic and fermionic estimators yielded similar distributions.
However, the fermionic estimator \eqref{fermionic particle number} seems to be positive-definite in tests at small particle number, while the bosonic estimator is not.
Therefore, especially at small particle number, the bosonic method may have much greater variance.

\subsection{The Contact}\label{sec:contact}

The contact is given by~\cite{PhysRevA.86.013626}
\begin{align}
	\hat{C} &= \frac{2 \pi M}{\hbar^2} \frac{dH}{d\log a}
	&
	\hat{C}\Delta x^2 &= 2\pi \frac{d\tilde{H}}{d\log a} 
	\label{eq:contact}
\end{align}
\Refs{Froehlich2011,PhysRevLett.109.130403} report a measurement of the contact.
Specifically they measure $C/k_F^2$ as a function of $1/\log k_F a_{2D}$ in the Fermi liquid regime.
In \Ref{PhysRevLett.109.130403} Fig. 4 they show a comparison with a zero-temperature fixed-node diffusion Monte Carlo calculation~\cite{PhysRevLett.106.110403}; the theory and experiment agree at $1/\log k_F a \gtrsim 0.5$ but the experiment and the theory diverge deeper into the weak-coupling regime $1/\ln k_F a \sim 0.3$.
The disagreement is only one data point, so who knows; the experimentalists did a ladder approximation and reproduced that one point.
I can't make heads or tails of the method; \Ref{doi:10.1063/1.443766} claims the solution is exact in the region bounded by the nodes.

\subsection{Vortex-Sensitive Observables}\label{sec:vortex}

The BKT transition is a transition between no vortices (at low temperature) and lots of vortex/antivortex pairs (at high temperature), which is driven by the entropic considerations of where all the vortices can be~\cite{1971JETP...32..493B,1972JETP...34..610B,Kosterlitz:1973}.
\Ref{PhysRevLett.129.076403} provides an analysis of the BKT transition by studying the condensate fraction [their (2)] and the Cooper pair momentum distribution [their (3)].
This transition is topological, in the sense that there isn't a good local order parameter to capture the change we're talking about.
Consider, for example, an observable that measures the local vorticity, like
\begin{align}
	\omega^3_x = \text{curl}(j)_x = \grad \times \vec{j}_x = \epsilon^{3mn} \partial^m j^n_x.
\end{align}
By periodic boundary conditions the net vorticity vanishes
\begin{align}
	\sum_x \omega^3_x = 0
	\label{eq:net vorticity}
\end{align}
if the current $j$ corresponds obeys a lattice continuity equation, since particles cannot leave across a boundary without coming back `on the other side'.
(The vorticity is entirely in the 3 direction, since the fermions are confined to 2D; in what follows we suppress the 3 superscript.)
This vanishes exactly because of the periodic boundary conditions; the boundary-condition-dependence is a hint it's accessing something topological.

Let us find the needed $j$, first in dimensionful language.
One well-defined method is to require the lattice continuity equation $\partial_t n = -\grad \cdot \vec{j}$ and to replace the left-hand side with $+i[H, n]$,
\begin{align}
	M\grad\cdot\vec{j}_z = -M \partial_t n_x = -i[MH, n_z]
	=
	-i \left[
	\sum_{xy}\Delta x^4 \psi\adjoint_x \left(\kappa_{xy} - \frac{M g}{2 \Delta x^2} \frac{\delta_{xy}}{\Delta x^2}\right) \psi_y,
	n_z\right]
\end{align}
because the interaction depends only on $n$s, which commute.  Moreover, the $g$-dependent piece gives a term proportional to $n$, so we need only worry about the commutator with the hopping.
Fourier transforming the $z$ coordinate and suppressing spin indices,
\begin{align}
	\sum_z \Delta x^2 e^{-ip \cdot z} (M i\grad \cdot \vec{j}_z)
	&=
	\left[
	\sum_{xy}\Delta x^4 \psi\adjoint_x \kappa_{xy} \psi_y,
	\sum_z \Delta x^2 e^{-ip \cdot z} n_z\right]
\nonumber\\
	\sum_z \Delta x^2 e^{-ip \cdot z} M i\grad \cdot \oneover{\Volume \Delta x^2} \sum_q e^{i q\cdot z} \vec{j}_q
	&=
	\sum_{xyz} \Delta x^6 e^{-ip \cdot z} \kappa_{xy} \left[ \psi\adjoint_x \psi_y, n_z\right]
\nonumber\\
	-\sum_z \Delta x^2 e^{-ip \cdot z} \oneover{\Volume \Delta x^2} \sum_q M \vec{q} \cdot \vec{j}_q e^{i q\cdot z} 
	&=
	-\sum_{xyz} \Delta x^6 e^{-ip \cdot z} \kappa_{xy} \left[n_z, \psi\adjoint_x \psi_y\right]
\nonumber\\
	-\oneover{\Volume} \sum_q M \vec{q} \cdot \vec{j}_q \sum_z e^{i(q-p) \cdot z} 
	&=
	-\sum_{xyz} \Delta x^6 e^{-ip \cdot z} \kappa_{xy} \left(\left[n_z, \psi\adjoint_x\right]\psi_y + \psi\adjoint_x\left[n_z, \psi_y\right]\right)
\nonumber\\
	\oneover{\Volume} \sum_q M \vec{q} \cdot \vec{j}_q \Volume \delta_{pq}
	&=
	\sum_{xyz} \Delta x^6 e^{-ip \cdot z} \kappa_{xy} \left(\psi\adjoint_x\delta_{xz} \psi_y - \psi\adjoint_x \psi_y \delta_{yz}\right) \Delta x^{-2}
\nonumber\\
	M \vec{p} \cdot \vec{j}_p
	&=
	\sum_{xz} \Delta x^4 e^{-ip \cdot z} \left(\psi\adjoint_z \kappa_{zx} \psi_x - \psi\adjoint_x \kappa_{xz} \psi_z\right)
\nonumber\\
	&=
	\oneover{\Volume \Delta x^2}\sum_{qxz} \frac{q^2}{2} \Delta x^4 e^{-ip \cdot z} \left(\psi\adjoint_z e^{-iq(z-x)} \psi_x - \psi\adjoint_x e^{-iq(x-z)} \psi_z\right)
\nonumber\\
	&=
	\oneover{\Volume \Delta x^2}\sum_{q} \frac{q^2}{2} \left(\sum_{xz} \Delta x^4 \psi\adjoint_z e^{-i(q+p)z+iqx} \psi_x - \sum_{xz} \Delta x^4 \psi\adjoint_x e^{-i(p-q)z-iqx)} \psi_z\right)
\nonumber\\
	&=
	\oneover{\Volume \Delta x^2}\sum_{q} \frac{q^2}{2} \left(\psi\adjoint_{-(p+q)} \psi_{-q} - \psi\adjoint_{-q} \psi_{p-q} \right)
\nonumber\\
	&=
	\half \oneover{\Volume \Delta x^2}\sum_{q} \left(q^2\psi\adjoint_{-(p-q)} \psi_q - q^2 \psi\adjoint_{q} \psi_{p+q} \right)
\nonumber\\
	&=
	\half \oneover{\Volume \Delta x^2}\sum_{q} \left((p+q)^2\psi\adjoint_{q} \psi_{p+q} - q^2 \psi\adjoint_{q} \psi_{p+q} \right)
\nonumber\\
	&=
	\half \oneover{\Volume \Delta x^2}\sum_{q} (p^2 + 2pq)\psi\adjoint_{q} \psi_{p+q}
\nonumber\\
	M \vec{p} \cdot \vec{j}_p
	&=
	\vec{p} \cdot \left(\half \oneover{\Volume \Delta x^2} \sum_q (\vec{p}+2\vec{q}) \psi\adjoint_q\psi_{p+q}\right)
\end{align}
where we leveraged the anticommutation rules, the Fourier transform conventions, the ability to shift the sum over the Brillouin zone without causing problems.
Since this is true for any mommentum $p$,
\begin{align}
	M \vec{j}_p = \half\oneover{V\Delta x^2} \sum_q (\vec{p}+2\vec{q}) \psi\adjoint_{q} \psi_{p+q}.
\end{align}
where the sum is over the Brillouin zone and both $p+q$ and $p+2q$ are understood mod the Brillouin zone.
We can now find $\omega$ as a function of momentum.
Fourier transforming,
\begin{align}
	\omega^3_k
	= \sum_x \Delta x^2 e^{-ikx} \omega^3_x
	= \sum_x \Delta x^2 e^{-ikx} \grad \times \vec{j}_x
	&= \sum_x \Delta x^2 e^{-ikx} \grad \times \oneover{\Volume \Delta x^2} \sum_p e^{+ipx} \vec{j}_p
\nonumber\\
	&= \sum_p i\vec{p} \times \vec{j}_p \oneover{\Volume} \sum_x e^{i(p-k)x}
	= \sum_p i\vec{p} \times \vec{j}_p \delta_{pk}
	= i \vec{k} \times \vec{j}_k
\end{align}
which shows that the total vorticity is 0.
Moreover, when calculating the vorticity we can drop the piece proportional to the momentum itself,
\begin{align}
	\omega^3_k
	= i \vec{k} \times \left(\vec{j}_k = \half\oneover{M \Volume\Delta x^2} \sum_q (\vec{k}+2\vec{q}) \psi\adjoint_{q} \psi_{k+q}\right)
	= i \half\oneover{M \Volume\Delta x^2} \sum_q \left(\vec{k} \times (\vec{k}+2\vec{q})\right) \psi\adjoint_{q} \psi_{k+q}
\nonumber\\
	= \oneover{M \Volume\Delta x^2} \sum_q (i\vec{k} \times \vec{q}) \psi\adjoint_{q} \psi_{k+q}
\end{align}


Instead of the total vorticity \eqref{net vorticity}, which must vanish by periodic boundary conditions, consider the vorticity-vorticity two-point correlation
\begin{align}
	\Omega_{\delta x} &= \oneover{\Volume \Delta x^2} \sum_x \Delta x^2 \Omega_{x,x-\delta x}
&	\Omega_{xy} &= \omega\adjoint_x \omega_y.
\end{align}
If there were one vortex/antivortex pair that were well-separated the correlations would reach far as a function of the separation $\delta$.
But when there are very many vortices and antivortices $\Omega$ should decay quickly.
Evaluated via convolution, one finds
\begin{align}
	\Omega_{\delta x}
	= \oneover{M^2}\left(\oneover{\Volume \Delta x^2}\right)^2 \sum_p e^{-ip \cdot \delta x} \omega\adjoint_p \omega_p
	= \oneover{M^2}\left(\oneover{\Volume \Delta x^2}\right)^4 \sum_{pkq} e^{-ip \cdot \delta x} (\vec{p}\times\vec{q})\cdot(\vec{p}\times\vec{k}) \psi\adjoint_{p+q}\psi_q\psi\adjoint_k\psi_{p+k}
\end{align}
where the dot product between the cross products arises naturally from considering $\omega\adjoint \cdot \omega$ (and we focus on the $z$ direction when the momenta are all confined to the plane).
Using a vector identity one finds
\begin{align}
	\Omega_{\delta x}
	= \oneover{M^2}\left(\oneover{\Volume \Delta x^2}\right)^4 \sum_{pkq} e^{-ip \cdot \delta x} \left[p^2 (k\cdot q) - (p\cdot k)(p\cdot q)\right] \psi\adjoint_{p+q}\psi_q\psi\adjoint_k\psi_{p+k}
\end{align}
with no need to subtract a disconnected piece; that piece is 0 by translation invariance, since $\expectation{\omega_x} = \oneover{\Volume} \expectation{\sum_x \omega_x} = 0$.
