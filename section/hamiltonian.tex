\section{Hamiltonian}\label{sec:hamiltonian}

Consider in two spatial dimensions the Hamiltonian
\begin{align}
	H
	&= \int d^2x\; \frac{\grad \psi\adjoint \cdot \grad \psi}{2M} + \int d^2x\; d^2y\; n_1(x) V(x,y) n_2(y)
	&
	n_i(x) &= \psi\adjoint_i(x) \psi_i(x)
	\\
	&= \int d^2x\; \frac{\grad \psi\adjoint \cdot \grad \psi}{2M} + g n_1(x) n_2(x)
\end{align}
if the interaction is $V(x,y) = -g \delta(x-y)$.

Let us agree to rescale all energies by multiplying by $M$ so that we only need to track dimensions of length,
\begin{align}
	HM
	&= \int d^2x\; \half \grad \psi\adjoint \cdot \grad \psi + \int d^2x\; d^2y\; n_1(x) (MV(x,y)) n_2(x)
	\\
	&= \int d^2x\; \half \grad \psi\adjoint \cdot \grad \psi + gM n_1(x) n_2(x)
\end{align}
So, `energies' have just dimensions of $[p^2] = -2$, so that from the first term $[\psi\adjoint\psi] = -2$ dimensions of length.
Then, looking at the right term we find that $[gM] = 0$, the coupling is dimensionless.
This is VERY different from 3 dimensions!

Let's take a square of side $L$ with periodic boundary conditions and divide it into $N_x$ sites on a side separated by $\Delta x = L/N_x$ so that the number of sites, the dimensionless volume $\Volume=N_x^2$.
If we pick $N_x$ odd then the sites take two integer coordinates from $-(N_x-1)/2$ to $+(N_x-1)/2$ and the dimensionless momenta $p$ are $2\pi / N_x$ times that same span.
The dimensionful coordinates get a $\Delta x$ upstairs, the dimensionful momenta a $\Delta x$ downstairs.
Sometimes we label momenta by $\vec{n}$, the integers that determine $p$ (or $n$ when the meaning is clear).

In dimensionful coordinates the Hamiltonian (meaning $HM$) is
\begin{align}
	MH &= \sum_{xy}\Delta x^4 \psi\adjoint_x \left(\kappa_{xy} - \frac{M\mu_0}{2} \frac{\delta_{xy}}{\Delta x^2} \right)\psi_y + \frac{M}{2} \sum_{xy} \Delta x^4 n_x V_{xy} n_y
	\nonumber\\
	&= 
	\sum_{xy}\Delta x^4 \psi\adjoint_x \left(\kappa_{xy} - \frac{M g}{2 \Delta x^2} \frac{\delta_{xy}}{\Delta x^2}\right) \psi_y
	+ \frac{M}{2} \sum_{xy} \Delta x^4 n_x \left( \frac{g}{\Delta x^2} \LegoSphere{0}_{xy} + \frac{C_{\vec{R}}}{\Delta x^2} \LegoSphere{\vec{R}}_{xy} \right) n_y
\end{align}
where $\LegoSphere{\vec{r}}$ is a lego sphere, described below, we pick $\mu_0 = -g \Delta x^{-2}$ to ensure we recover the correct Hamiltonian in the continuum limit, we take the dimensions out of the potential, since $g\delta(x-y) \goesto g \Delta x^{-2} \delta_{xy}$.
We use the perfect dispersion relation for $\kappa$, the number operator $n$, with dimensionful field operators,
\begin{align}
    \kappa_{xy} =& \oneover{\Volume \Delta x^2} \sum_{p} \frac{p^2}{2} e^{-i p \cdot (x-y)}
	&
	n_x =& \psi\adjoint_x \psi_x
	&
	\{\psi_{x,\sigma}, \psi\adjoint_{y,\tau} \} =& \frac{\delta_{xy}}{\Delta x^2} \delta_{\sigma\tau},
\end{align}
where $n$ contains an implicit sum over spins.
\Tabref{ft} contains a summary of our Fourier transformation conventions.

\begin{table}
\large
\begin{tabular}{R|L|L}
\text{Infinite Volume}								&	\text{Continuum}
													&	\text{Lattice}
\\[0.5em]\hline
\psi(x) 											=& \int \frac{d^d p}{(2 \pi)^d} ~ e^{+i p\cdot x} ~ \psi(p)
													 & \int_{-\pi/\Delta x}^{\pi/\Delta x} \frac{d^d p}{(2\pi)^d} ~ e^{i p\cdot x} ~ \psi(p)
\\[0.5em]
\psi(p) 											=& \int d^d x ~ e^{-i p\cdot x} ~ \psi(x)
													 & \sum_x \Delta x^d ~ e^{-i p\cdot x} ~ \psi(x) 
\\[0.5em]
\int \frac{d^d p}{(2 \pi)^d} ~ e^{+i p\cdot (x-x')}	=& \delta^d(x-x')
													 & \frac{\delta_{x x'}}{\Delta x^d} 
\\[0.5em]
(2\pi)^d\delta^d(p-p') 								=& \int d^d x ~ e^{-i (p - p') \cdot x}
													 & \sum_x \Delta x^d ~ e^{-i (p - p') \cdot x}
\\[0.5em]\hline
\text{Finite Volume}								&	\text{Continuum}
													&	\text{Lattice}
\\[0.5em]\hline
\psi(x)												=& \frac{1}{V} \sum_p e^{+i p\cdot x} ~ \psi(p)
													 & \frac{1}{(N_x \Delta x)^d} ~ \sum_p e^{+i p\cdot x} ~ \psi(p)
\\[0.5em]
\psi(p)												=& \int_V d^d x ~ e^{-i p\cdot x} ~ \psi(x)
													 & \sum_x \Delta x^d ~ e^{-i p\cdot x} ~ \psi(x)
\\[0.5em]
\sum_p ~ e^{+i p\cdot (x-x')}						=& V \delta^d(x-x')
													 & (N_x \Delta x)^d \frac{\delta_{x x'}}{\Delta x^d}
\\[0.5em]
\delta_{pp'}										=& \frac{1}{V} \int_V d^d x ~ e^{-i (p - p') \cdot x}
													 & \frac{1}{(N_x \Delta x)^d} \sum_x \Delta x^d e^{-i (p-p')\cdot x}
\\[0.5em]\hline
\text{Dimensionless}								&
													&	\text{Lattice}
\\[0.5em]\hline
\psi_a												=&& \oneover{N_x^d} \sum_{k} e^{+2\pi i k\cdot a/N_x} \psi_k
\\[0.5em]
\psi_k												=&& \sum_a e^{-2\pi i k \cdot a / N_x} \psi_a
\\[0.5em]
\sum_{k} e^{+2\pi i k\cdot(a-a') / N_x}				=&& N_x^d \delta_{aa'}
\\[0.5em]
\delta_{kk'}										=&& \oneover{N_x^d} \sum_a e^{-2\pi i (k-k')\cdot a /N_x}
\end{tabular}
\caption{
	A complete summary of consistent Fourier conventions with dimensionful variables.
	The infinite continuum volume is $\Reals^d$ while the periodic finite volume is given by a $d$-dimensional cube with length $L$ and volume $V=L^d$.
	The lattice has spacing $\Delta x$; in finite-volume there are $N_x$ sites to a side, $L=N_x \Delta x$.
	In a finite volume momenta are integer multiples of $2\pi/L$.
	In these conventions the dimensionless coordinates can be found from the finite-volume lattice language by sending $\Delta x\goesto 1$ and rewriting momenta as multiples of $2\pi/N_x (\Delta x=1)$.
	(You can use dimensionless operators independently of the choice to use dimensionless coordinates).
}
\label{tab:ft}
\end{table}


Passing to dimensionless variables,
\begin{align}
	x & \goesto a \Delta x
	&
	p &\goesto \frac{2\pi k}{N_x \Delta x}
	&
	\psi_x & \goesto \tilde{\psi}_a / \Delta x
	&
	n &\goesto \tilde{n} / \Delta x^2
\end{align}
where $a$ and $k$ are triplets of integers that span from $-(N_x-1)/2$ to $+(N_x-1)/2$, we find
\begin{align}
	MH &= 
		\sum_{ab}\Delta x^4 \frac{\tilde{\psi}\adjoint_a}{\Delta x} \left(
				\oneover{\Volume \Delta x^2} \sum_{k} \frac{(2\pi k)^2}{2 \Volume \Delta x^2} e^{-2\pi i k \cdot(a-b)/N_x} 
				- \frac{Mg}{2\Delta x^2} \frac{\delta_{ab}}{\Delta x^2}
			\right) \frac{\tilde{\psi}_b}{\Delta x}
		+ \frac{M}{2} \sum_{a} \Delta x^4 \frac{\tilde{n}_a}{\Delta x^2} \left(
				\frac{g}{\Delta x^2} \LegoSphere{0}_{ab}
				+ \frac{C_{\vec{R}}}{\Delta x^2} \LegoSphere{\vec{R}}_{ab}
			\right) \frac{\tilde{n}_b}{\Delta x^2}
	\\
	MH &=
		\sum_{ab}\frac{\tilde{\psi}\adjoint_a}{\Delta x} \left(
			\oneover{\Volume} \sum_{k} \frac{(2\pi k)^2}{2 \Volume} e^{-2\pi i k \cdot(a-b)/N_x} 
			- \frac{gM}{2} \delta_{ab}
			\right) \frac{\tilde{\psi}_b}{\Delta x}
		+ \frac{M}{2} \sum_{ab} \tilde{n}_a \left(
				\frac{g}{\Delta x^2} \LegoSphere{0}_{ab}
				+ \frac{C_{\vec{R}}}{\Delta x^2} \LegoSphere{\vec{R}}_{ab}
			\right) \tilde{n}_b
	\\
	MH \Delta x^2 &=
		\sum_{ab} \tilde{\psi}\adjoint_a \left(
			\oneover{\Volume} \sum_{k} \frac{(2\pi k)^2}{2 \Volume} e^{-2\pi i k \cdot(a-b)/N_x} 
			- \frac{gM}{2} \delta_{ab}
			\right) \tilde{\psi}_b
		+ \frac{1}{2} \sum_{ab} \tilde{n}_a \left(
				gM \LegoSphere{0}_{ab}
				+ C_{\vec{R}} M \LegoSphere{\vec{R}}_{ab}
			\right) \tilde{n}_b,
\end{align}
so that we can define $\tilde{V}_{ab} = C_{\vec{R}} \LegoSphere{\vec{R}}_{ab}$ with an implied sum over the Lego sphere radii $\vec{R}$ we support; $C_0 = g$.

The right-hand side is now completely dimensionless, depending only on the coefficients $C$ and the number of sites $N_x$; no dimensionful scale remains.
This leads to some magic.
If we fix $C$ and $N_x$, it does not matter what actual, physical $\Delta x$ is---the spectra for all $\Delta x$s are related simply by a rescaling.
\emph{This is really different from \Refs{Alexandru:2019gmp,Alexandru:2020zti}, where the kinetic and potential terms scale differently with $\Delta x$!}
In those cases, if you change $\Delta x$ while holding $g$ and $N_x$ fixed the spectra do not just perfectly scale.
The difference is that in $d$ dimensions a contact interaction is like $\Delta x^{-d}$; in 3D this is \emph{more} divergent than the kinetic term, while in 2D it's a perfect match.

So, we can work with the dimensionless
\begin{align}
	\tilde{H} = MH\Delta x^2 &= 
		\sum_{ab} \tilde{\psi}\adjoint_a \left(
			\tilde{\kappa}_{ab}
			- \frac{\tilde{C}_0}{2} \delta_{ab}
			\right) \tilde{\psi}_b
		+ \frac{1}{2} \sum_{ab} \tilde{n}_a \tilde{V}_{ab} \tilde{n}_b
\\
	\tilde{\kappa}_{ab} &=
				\oneover{\Volume} \sum_{k} \frac{(2\pi k)^2}{2 \Volume} e^{-2\pi i k \cdot(a-b)/N_x} 
\\
	\tilde{V}_{ab} &= 
				gM \LegoSphere{0}_{ab}
				+ C_{\vec{R}} M \LegoSphere{\vec{R}}_{ab}.
				=
				\tilde{C}_0 \LegoSphere{0}_{ab}
				+ \tilde{C}_{\vec{R}} \LegoSphere{\vec{R}}_{ab};
\label{eq:dimensionless potential}
\end{align}
and Newton's 3rd law means $\tilde{V}=\tilde{V}\transpose$.

According to \Ref{Korber:2019cuq} the continuum tuning is given by
\begin{align}
	\tilde{C}_0 = gM &= - \frac{2\pi}{\log(A \Lambda)}
	\label{eq:tuning}
\end{align}
where $A$ is explained below \eqref{no-constant}.
The conclusion there is that $A$ is naturally of size $\sim L$; since $\Lambda\sim \Delta x^{-1}$ the log will be of something like $N_x$.
I'll work this out below after explaining wtf $A$ is.
But the punchline is that indeed the tuning \emph{also} only depends on dimensionless things.
So it's really true; our simulations can be entirely described in terms of the dimensionless $gM$ and $N_x$ and that is all.

\subsection{$D_4$ Symmetry and Lego Spheres}

The finite volume / Brillouin zone has the symmetry of a square, $D_4$, the group of 8 elements (the identity $E$, 3 nontrivial rotations $2C_4(z)$ and $C_2(z)$, 2 edge $2C'_2$ and 2 vertex $2C''_2$ reflections).
Some details are shown in \Tabref{D4}.
The `$S$-wave' is the $A_1$ representation.

\begin{table}
    \begin{tabular}{C|RRRRR}
        D_4   &   E   &   2C_4(z)   &   C_2(z)    &   2C'_2 &   2C''_2    \\\hline
        A_1   &   1   &   1         &   1         &   1     &   1         \\
        A_2   &   1   &   1         &   1         &   -1    &   -1        \\
        B_1   &   1   &   -1        &   1         &   1     &   -1        \\
        B_2   &   1   &   -1        &   1         &   -1    &   1         \\
        E     &   2   &   0         &   -2        &   0     &   0
    \end{tabular}
    \hspace{3em}
    \begin{tabular}{C|CCCCC}
        \otimes &   A_1   &   A_2 &   B_1 & B_2   &   E   \\\hline
        A_1     &   A_1   &   A_2 &   B_1 & B_2   &   E   \\
        A_2     &   A_2   &   A_1 &   B_2 & B_1   &   E   \\
        B_1     &   B_1   &   B_2 &   A_1 & A_2   &   E   \\
        B_2     &   B_2   &   B_1 &   A_2 & A_1   &   E   \\
        E       &   E     &   E   &   E   & E     &   A_1 \oplus A_2 \oplus B_1 \oplus B_2   \\
    \end{tabular}
    \caption{
        Left: the character table of $D_4$.
        Right: the product table for representations.
    }
    \label{tab:D4}
\end{table}

Translationally invariant potentials $\tilde{V}_{ab}$ can be expressed as $\tilde{V}_{a,a+r} = \tilde{V}_{0,r}$  in terms of the relative lattice coordinate $r$.
We can construct `rotationally'-invariant (meaning: transforms like the $A_1$ representation of $D_4$) interaction potentials $V^R_{0,r}$ out of `Lego spheres' \LegoSphere{R} that are obviously $A_1$ interactions that depend on a `radius' $R$,
\begin{align}
    \tilde{V}_{0,r} &= \sum_{R} C_R \LegoSphere{R}_{0,r}
    &
    \LegoSphere{R}_{0,r} &= \oneover{\shell{R}^2}\sum_{g \in D_4} \delta_{r, gR}
    &
    \LegoSphere{R}_{ab} &= \oneover{\shell{R}^2}\sum_{g \in D_4} \delta_{b-a,gR}
    \label{eq:Lego sphere}
\end{align}
where the normalization $\shell{R}$ is given by $\sqrt{\textrm{the number of sites in the sphere}}$ and $C_R$ is the Wilson coefficient / strength of the interaction for that shell.
{\bf In our 3D convention we left out the normalization} but I think it's beneficial to put it in, because it's as though we split the $R=\vec{0}$ contact operator into $\shell{R}^2$ pieces, the `surface area' of the Lego sphere.
We have to track the whole vector radius rather than just its magnitude, since some sites the same distance from the origin can never be mapped into one another under a $D_4$ operation, such as $(3,4)$ and $(0,5)$; even though these radii have the same magnitude, they belong to different operators which need not talk to one another or have similar coefficients.
However, the Lego spheres built from vectors which appear in the same orbit of $D_4$, such as $(3,4)$ and $(4,-3)$, generate the same Lego sphere.
So, we can conventionally pick one representative radius vector with non-negative components.

We can therefore write a fully general translationally-invariant $A_1$ potential as
\begin{align}
	\tilde{V}_{ab} &= \sum_{R} \tilde{C}_R \LegoSphere{R}_{ab}.
\end{align}
The simplest choice is $\tilde{C}_{\vec{0}} = gM$ and all other $\tilde{C}_R=0$.
In 3D we observe that this choice gets linear discretization errors (presumably in this case that means linear in $N_x\inverse$).
With additional supported Lego spheres presumably this improves.
