\section{Hamiltonian}\label{sec:hamiltonian}

Consider in two spatial dimensions the Hamiltonian
\begin{align}
	H
	&= \int d^2x\; \frac{\grad \psi\adjoint \cdot \grad \psi}{M} - \int d^2x\; d^2y\; \left[\psi\adjoint_1(x) \psi_1(x)\right] V(x,y) \left[\psi\adjoint_2(y) \psi_2(y)\right]
	\\
	&= \int d^2x\; \frac{\grad \psi\adjoint \cdot \grad \psi}{M} - g \left[\psi\adjoint_1(x) \psi(x)\right] \left[\psi\adjoint_2(y) \psi(y)\right]
\end{align}
if the interaction is $V(x,y) = g \delta(x-y)$.

Let us agree to rescale all energies by multiplying by $M$ so that we only need to track dimensions of length,
\begin{align}
	HM
	&= \int d^2x\; \grad \psi\adjoint \cdot \grad \psi - gM \left[\psi\adjoint_1(x) \psi(x)\right] \left[\psi\adjoint_2(y) \psi(y)\right]
\end{align}
So, `energies' have just dimensions of $[p^2] = -2$, so that from the first term $[\psi\adjoint\psi] = -2$ dimensions of length.
Then, looking at the right term we find that $[gM] = 0$, the coupling is dimensionless.
This is VERY different from 3 dimensions!

Let's take a square of side $L$ with periodic boundary conditions and divide it into $N_x$ sites on a side separated by $\Delta x = L/N_x$ so that the number of sites, the dimensionless volume $V=N_x^2$.
If we pick $N_x$ odd then the sites take an integer label from $-(N_x-1)/2$ to $+(N_x-1)/2$ and the dimensionless momenta $p$ are $2\pi / N_x$ times that same span.

The Hamiltonian (meaning $HM$) is
\begin{align}
	MH &= \sum_{xy} \psi\adjoint_x \kappa_{xy} \psi_y - \frac{gM}{\Delta x^2} \sum_{x} n^1_x n^2_x
	&
	\kappa_{xy} &= \oneover{V} \sum_{p} \frac{p^2}{2\Delta x^2} e^{-i p \cdot (x-y)}
	&
	n_x &= \psi\adjoint_x \psi_x
\end{align}
for spin species 1 and 2 and dimensionless lattice $p$ and $x$.

But \emph{now comes the magic}.
Both terms have $\Delta x^2$ downstairs.
In 3D the kinetic piece has $\Delta x^2$ and the potential $\Delta x^3$.
So now we can go `truly dimensionless', and clear the $\Delta x^{-2}$ from the right,
\begin{align}
	\tilde{H} = HM \Delta x^2
	&=
	\sum_{xy} \psi\adjoint_x k_{xy} \psi_y - gM \sum_x n^1_x n^2_x
	&
	k_{xy} &= \oneover{V} \sum_{p} \frac{p^2}{2} e^{-i p \cdot (x-y)}
	\label{eq:hamiltonian}
\end{align}
The right-hand side now depends \emph{only} on the number of sites $N_x$; no dimensionful scale remains.
If we increase $L$ and $\Delta x$ holding $N_x$ fixed we get the same right-hand side.

Suppose we find the spectrum for a particular $gM$ and $N_x$.
We can interpret it in two ways:
\begin{enumerate}
	\item First, we can think of fixing $L$, which determines $\Delta x$.
	\item Or, we can think of fixing $\Delta x$, which determines $L$.
\end{enumerate}
In other words, you can use the same exact data and use it to think about the infinite-volume or continuumn limits.

Is that really true?  I suppose it depends on the tuning of $gM$.
According to K\"{o}rber, Berkowitz, and Luu the tuning is
\begin{align}
	gM &= - \frac{2\pi}{\log(\tilde{a} \Lambda)}
	\label{eq:tuning}
\end{align}
where $\tilde{a}$ is explained below.
The conclusion there is that $\tilde{a}$ is naturally of size $\sim L$; since $\Lambda\sim \Delta x^{-1}$ the log will be of something like $N_x$.
I'll work this out below after explaining wtf $\tilde{a}$ is.
But the punchline is that indeed the tuning \emph{also} only depends on dimensionless things.
So it's really true; our simulations can be entirely described in terms of the dimensionless $gM$ and $N_x$ and that is all.
