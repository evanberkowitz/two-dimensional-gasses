\section{Sampling Algorithms}\label{sec:sampling}

We can sample our partition function of auxiliary fields \eqref{computational partition function} however we wish.

The classic LQCD approach is HMC.
Another approach we would like to try is based on normalizing flows.

\subsection{HMC}

Hybrid Monte Carlo~\cite{Duane1987}, HMC, (sometimes called `Hamiltonian Monte Carlo') is a LQCD workhorse.
It proceeds as follows:

\begin{enumerate}
	\item \todo{explain HMC}.
\end{enumerate}

The most straightforward application is to the $\vec{h}=0$ case where $\det \dd$ is a square and we can construct a pseudofermion method.
However, even when $\vec{h}\neq0$ we can still differentiate $S$: here is where the authomatic differentiation of \pytorch really shines: an implementation of $S$ automatically provides an implementation of $\partial S/\partial A$.

\subsection{HMC with Learning}

Machine learning methods can be inserted into HMC to help explore the configuration space more rapidly~\cite{Foreman:2021ljl}.
One promising approach is to replace the leapfrog integration with a fast, learned network with a simple Jacobian~\cite{Foreman:2021rhs}

\subsection{Flow-Based Sampling}

An exciting alternative to HMC is flow-based sampling~\cite{Albergo:2019eim,Rezende:2020hrd,Kanwar:2020xzo}, which has found success in a variety of lattice quantum field theory settings~\cite{Albergo:2022qfi}.
\Ref{Albergo:2021vyo} provides an end-to-end example as an interactive Jupyter notebook.
