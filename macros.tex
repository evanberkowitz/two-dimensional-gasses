\usepackage{xspace}
\usepackage{bbm}
\usepackage{simplewick}

%%%%
%%%%    Project Specifics
%%%%

\newcommand{\repoURL}{https://github.com/evanberkowitz/two-dimensional-gasses}

%%%%
%%%%    lower-case caligraphic \mathpzc font
%%%%

\DeclareFontFamily{OT1}{pzc}{}
\DeclareFontShape{OT1}{pzc}{m}{it}{<-> s * [1.10] pzcmi7t}{}
\DeclareMathAlphabet{\mathpzc}{OT1}{pzc}{m}{it}

%%%%
%%%%    Some few-letter shortcuts
%%%%

\newcommand{\BB}{\ensuremath{\mathbb{B}}\xspace}
\newcommand{\dd}{\ensuremath{\mathbbm{d}}\xspace}
\newcommand{\DD}{\ensuremath{\mathbb{D}}\xspace}
\newcommand{\FF}{\ensuremath{\mathbb{F}}\xspace}
\newcommand{\G}{\ensuremath{\mathcal{G}\xspace}}
\newcommand{\Ham}{\ensuremath{\mathcal{H}}\xspace}
\newcommand{\MM}{\ensuremath{\mathbb{M}}\xspace}
\newcommand{\p}{\ensuremath{\mathpzc{p}}\xspace}
\newcommand{\PP}{\ensuremath{\mathbb{P}}\xspace}
\newcommand{\uu}{\ensuremath{\mathbbm{u}}\xspace}
\newcommand{\UU}{\ensuremath{\mathbb{U}}\xspace}
\renewcommand{\O}{\ensuremath{\mathcal{O}}\xspace}
\newcommand{\dbar}{\mathchar'26\mkern-9mu d}

\newcommand{\expectation}[1]{\ensuremath{\left\langle #1 \right\rangle}}

%%%%
%%%%    Referring to Parts of the Document
%%%%

\newcommand{\secref}[1]{Sec.~\ref{sec:#1}}
\newcommand{\Secref}[1]{Section~\ref{sec:#1}}
\newcommand{\appref}[1]{App.~\ref{sec:#1}}
\newcommand{\Appref}[1]{Appendix~\ref{sec:#1}}
\newcommand{\tabref}[1]{Tab.~\ref{tab:#1}\xspace}
\newcommand{\Tabref}[1]{Table~\ref{tab:#1}\xspace}
\newcommand{\figref}[1]{Fig.~\ref{fig:#1}\xspace}
\newcommand{\Figref}[1]{Figure~\ref{fig:#1}\xspace}
\renewcommand{\eqref}[1]{(\ref{eq:#1})\xspace}
\newcommand{\Eqref}[1]{Equation~\ref{eq:#1}\xspace}
\newcommand{\R}[1]{Ref.~\cite{#1}\xspace}
\newcommand{\Reference}[1]{Reference~\cite{#1}}
\newcommand{\Refs}[1]{Refs.~\cite{#1}}
\newcommand{\References}[1]{References~\cite{#1}}

%%%%
%%%%    Referring to Parts of the GitHub Repository
%%%%

\newcommand{\todo}[1]{\textbf{\color{red}TODO: #1}}
\newcommand{\issue}[1]{\href{\repoURL/issues/#1}{Issue #1}}
\newcommand{\pullrequest}[1]{\href{\repoURL/pulls/#1}{Pull Request #1}}
\newcommand{\sourcefile}[1]{\href{\repoURL/blob/\gitBranch/python/tdg/#1}{#1}}
\newcommand{\incode}[2]{\href{\repoURL/blob/\gitBranch/python/tdg/#1}{#2}}

%%%%
%%%%    Referring to Other Documents
%%%%

\renewcommand{\doi}[1]{\href{http://doi.org/#1}{[#1]}}
\newcommand{\arxiv}[1]{\href{http://www.arxiv.org/abs/#1}{arXiv:#1}}

%%%%
%%%%    Mathematical Symbols
%%%%

\newcommand{\goesto}{\ensuremath{\rightarrow}}
\newcommand{\widevec}[1]{\ensuremath{\overrightarrow{#1}}}
\newcommand{\cross}{\ensuremath{\times}}
\newcommand{\infinity}{\infty}
\newcommand{\Integers}{\mathbb{Z}\xspace}
\newcommand{\integers}{\Integers}
\newcommand{\one}{\ensuremath{\mathbbm{1}}}
\newcommand{\two}{\ensuremath{\mathbbm{2}}}
\newcommand{\order}[1]{\ensuremath{\mathcal{O}\left(#1\right)}\xspace}
\newcommand{\Rationals}{\mathbb{Q}\xspace}
\newcommand{\Reals}{\mathbb{R}\xspace}
\newcommand{\union}{\ensuremath{\cup}}
\DeclareMathOperator{\erf}{erf}
\renewcommand{\mod}[1]{\ensuremath{\ \left(\text{mod }#1\right)}}
\newcommand{\grad}{\ensuremath{\nabla}}
\newcommand{\up}{\ensuremath{\uparrow}}
\newcommand{\down}{\ensuremath{\downarrow}}
\newcommand{\dn}{\down}
\newcommand{\shell}[1]{\ensuremath{\mathcal{N}_{#1}}}
\newcommand{\LegoSphere}[1]{\ensuremath{\mathcal{S}^{#1}}}
\newcommand{\Volume}{\ensuremath{\mathcal{V}}\xspace}
\newcommand{\doubleOccupancy}{\ensuremath{\textrm{do}}\xspace}
\newcommand{\DoubleOccupancy}{\ensuremath{\textrm{DO}}\xspace}
\newcommand{\binding}{\ensuremath{\mathcal{E}_B}\xspace}
\newcommand{\lpartial}{\ensuremath{\overleftarrow{\partial}}}

%%%%
%%%%    Hyperbolic Trig
%%%%

% Most are already available if you \usepackage{amsmath}.
% However, those below are missing

\DeclareMathOperator{\sech}{sech}
\DeclareMathOperator{\csch}{csch}
\DeclareMathOperator{\arccosh}{arccosh}
\DeclareMathOperator{\arcsinh}{arcsinh}
\DeclareMathOperator{\arctanh}{arctanh}
\DeclareMathOperator{\arcsech}{arcsech}
\DeclareMathOperator{\arccsch}{arccsch}
\DeclareMathOperator{\arccoth}{arccoth}

% Additionally, there are some missing trig functions:

\DeclareMathOperator{\arcsec}{arcsec}
\DeclareMathOperator{\arccot}{arccot}
\DeclareMathOperator{\arccsc}{arccsc}

%%%%
%%%%    Fractions
%%%%

\newcommand{\oneover}[1]{\ensuremath{\frac{1}{#1}}}                             %   1/[argument]
\newcommand{\inverse}{\ensuremath{^{-1}}}                                       %   argument^-1
\newcommand{\half}{\ensuremath{\frac{1}{2}} }                                   %   1/2
\newcommand{\quarter}{\ensuremath{\frac{1}{4}} }                                %   1/4


%%%%
%%%%    Mathematical Delimiters
%%%%
\newcommand{\abs}[1]{\ensuremath{\left| #1 \right|}\xspace}
\newcommand{\magnitude}{\abs}
\newcommand{\norm}[1]{\left\| #1 \right\|}
\newcommand{\average}[1]{\ensuremath{\left\langle #1 \right\rangle}\xspace}

% Quantum mechanics bra-ket notation:
\newcommand{\ket}[1]{\ensuremath{\left|\;#1\;\right\rangle}}
\newcommand{\bra}[1]{\ensuremath{\left\langle\;#1\;\right|}}
\newcommand{\bracket}[2]{\ensuremath{\left\langle\;#1\;\middle|\;#2\;\right\rangle}}
\let\braket\bracket
\newcommand{\matrixElement}[3]{\ensuremath{\left\langle\;#1\;\middle|\;#2\;\middle|\;#3\;\right\rangle}}


%%%%
%%%%    Matrices
%%%%

\newcommand{\identity}{\ensuremath{\mathds{1}}}
\newcommand{\diag}[1]{\ensuremath{\text{diag}\left(#1\right)}}
\newcommand{\tr}[1]{\ensuremath{\text{tr}\left[#1\right]}}
\newcommand{\trlog}[1]{\ensuremath{\text{tr log}\left[#1\right]}}
\newcommand{\transpose}{\ensuremath{{}^{\top}}}
\newcommand{\adjoint}{\ensuremath{{}^{\dagger}}}
\newcommand{\conjugate}{\ensuremath{{}^*}\xspace}

%%%%
%%%%    Math Tables
%%%%
\usepackage{array}
\newcolumntype{L}{>{$}l<{$}}
\newcolumntype{C}{>{$}c<{$}}
\newcolumntype{R}{>{$}r<{$}}


%%%%
%%%%    Physical Quantities and Constants
%%%%

\newcommand{\deBroglie}{\ensuremath{\lambda_{\text{dB}}}\xspace}
\newcommand{\entropy}{\ensuremath{\mathcal{S}}\xspace}


%%%%
%%%%    Software
%%%%

\newcommand{\bash}{\texttt{bash}\xspace}
\newcommand{\git}{\texttt{git}\xspace}
\newcommand{\make}{\texttt{make}\xspace}
\newcommand{\mpi}{\texttt{MPI}\xspace}
\newcommand{\python}{\texttt{python}\xspace}
\newcommand{\pytorch}{\texttt{pytorch}\xspace}

% Put an xspace after \LaTeX:
\let\builtinLaTeX\LaTeX
\def\LaTeX{\builtinLaTeX\xspace}
